\chapter{Animation}
\label{chapter5}

The animatable body exported by Blender is in Ogre xml mesh and skeleton format. They are converted to binary skeleton and mesh files by the OgreXML serializer, and imported natively into the software. Prior to animation, the bones are initialized to be oriented by the data from depth sensor. Afterwards, they are updated with the orientation data.

\section{Initialization}
\label{section5_1}

Animatable meshes are loaded and maintained via the SkeletalMesh class I have implemented, which bridges the gap between Ogre�s skeletal animating system and the input from Kinect sensor. When an instance of the object is initialized, it is given a mesh file. After the mesh is loaded with the skeleton data, the bone list is iterated through, given the initial orientation uniquely for each animatable bone with the Kinect sensor. 

The animatable bones are set to be manually controlled, not inherit orientation from parents, given the initial orientations. At that state, the bone is set to initial state, to be reset at every frame with updated orientations. The non-animated bones are left to be automatically controlled and inherit orientation in order to keep them aligned with their parent bones. 

\section{Animation}
\label{section5_2}
Every frame, the orientation information from the bones are extracted in 3x3 matrix form., along with the confidence of the sensor in that orientation. If the confidence is less than 0.5 in 1, the bones are left as they were in the previous frame to avoid unnatural movements. 

The matrix from the Kinect is row major, whereas Ogre works with column major matrices. The transpose matrix is given to Ogre, turned into a quaternion and converted to local coordinate space. The updated orientation is fed into the bone, after it is multiplied with the initial orientation. The root bone is translated in local space in the end, to simulate the translation, above rotation.

This technique is used with the top-part of the dress as well, and it will be used for jackets, shirts and other clothing ware which do not leave the body surface by a great distance. 

\section{Algorithm}
\label{section5_3}

I used linear weighted skin blending \cite{Kavan2003} in order to simulate deformation on the characters skin. The effects of four bones with different weights are combined linearly to change the positions and normal of vertices.

\subsection{The bone transformation Pseudo Code}

The pseudo code below is executed during the pre-render cycle. The bone
orientations are set to be ready for animation, deformation and rendering. At
every render cycle, update Skeleton function is called, which automatically
fetches the coordinates from the Kinect and sets up the skeleton for vertex
blending. 

\begin{algorithm}
\dontprintsemicolon % Some LaTeX compilers require you to use \dontprintsemicolon instead
\SetKwBlock{function}{function}{endfunction}
\function(transformBone\ArgSty{(bone)}){
\KwIn{A bone and the corresponding orientation matrix from Kinect}
\KwOut{The same bone with updated orientation}
$q_I =$ initial orientation of bone\;
$q_N =$ relative orientation\;
$q_K = 3\times3$ Orientation Matrix From Kinect\;
\If{$kinect_{confidence} > 0.5$} {
  $q_Q = toQuaternion(q_K)$\;
  $q_N = toLocalSpace(q_Q)$\;
  $q = q_N \times q_I$\;
  $bone.orientation=q.normalise{\left(\right)}$\; }}

\If{$user$ ${\bf is }$ $new$} {
  $p_{torso}.initialize()$\tcc*{Initialize torso position}  \;
 }
\ForEach {bone}{
\If{$bone_{orientation}$ ${\bf is }$ $new$}{
$transformBone(bone)$\;
$skeleton.needsUpdate()$\;}}
\caption{Bone transformation algorithm}
\label{algo:transformBone}
\end{algorithm}

\subsection{Deformation Pseudo Code}

Algorithm \ref{algo:updateMesh} is executed prior to rendering, to update the
vertices of the skin. The vertex blending function is where the deformation
actually occurs.

\begin{algorithm}
\dontprintsemicolon % Some LaTeX compilers require you to use \dontprintsemicolon instead
\SetKwBlock{function}{function}{endfunction}
\function(prepareBlendMatrices\ArgSty{(mesh)}){
\ForEach {bone}{
bone.applyScale()\;
bone.applyTransform()\;
bone.applyOrientation()\;
}
$i=0$\;
\ForEach {bone}{
$m[i] = bone.getTransformationMatrix4\times4$\;
$i++$\;
}
mapIndex=mesh.getIndexMap() \tcc*{Index Map contains the bone pointers for
every vertex}\;
$i=0$\;
\ForEach {indexSet ${\bf in }$ mapIndex}{
$m_b[i] = indexSet[i]+m$ \tcc*{Blend matrices are pointers to the individual
bone matrices}\;
$i++$\; }
\Return $m_b$\;
}

\function(vertexBlend\ArgSty{($m_b$)}){
pos=*mesh.positions\tcc*{Pointer to the position matrix}\;
norm=*mesh.normals\tcc*{Pointer to the normal matrix}\;
$b_i=$*mesh.blendIndices\tcc*{Pointer to the blend index matrix}\;
$b_w=$*mesh.blendWeights\tcc*{Pointer to the blend weight matrix}\;
\ForEach {4 vertices ${\bf in }$ pos}{
\ForEach {vertex ${\bf in }$ 4 vertices}{ 
$m[1,2,3,4]=m_b[b_i[vertex]]$ \tcc*{Weighting Bones}\;
$m_c[j]=collapseMatrix(m,b_w[vertex])$\;
}
$pos[4 vertex]=m_c\times pos[4 vertex]$
$norm[4 vertex]=m_c\times norm[4 vertex]$
}}
\If {skeleton needs update}{
$m_b=prepareBlendMatrices(skeleton.mesh)$\;
$vertexBlend(m_b)$\;
}
\caption{Mesh update algorithm called at every frame.}
\label{algo:updateMesh}
\end{algorithm}