\chapter{Non-Linear Finite Element Method}
\label{chapter5}

This chapter explains in detail the stages of the nonlinear FEM and a verification procedure for measuring the
correctness of the proposed nonlinear FEM.

\section{Non-Linear FEM using Tetrahedral Elements with Green-Lagrange Strains}
\label{sec4NonLinFEM}

Proposed system uses non-linear FEM due to accuracy reasons. In this chapter, algorithm of non-linear FEM solution, the development of Green-Lagrange strains (large deformation strains) $\eta$ that leads to non-linear FEM, stiffness matrix $K$ and the solution of the system with Newton-Raphson method will be explained. Our proposed non-linear FEM solution algorithm for 3D tetrahedral element consist of 4 main parts:

\begin{enumerate}
\item Tetrahedralization;
\item Construction of nonlinear elemental stiffness matrices;
\item Construction of nonlinear element residuals;
\item Solution of the non-linear system with Newton-Raphson method that gives unknown nodal displacements.
\end{enumerate}

The proposed nonlinear FEM uses linear FEM's style in the sense that it does not require the explicit use of weight functions, differential equations and integrals. Moreover, our approach extends the linear FEM to the nonlinear FEM by extending the linear
strains to the Green-Lagrange strains.

\subsection{Tetrahedralization}
We have the information of nodal positions and which nodes belong to which element from the tetrahedralization process like we did for linear FEM solution. We use nodal positions to construct elemental stiffness matrices, and element's nodal information to assemble the element's Jacobian matrices to form global Jacobian matrices and element's residuals to form global residual vectors for every step of Newton-Raphson process.


\subsection{Construction of Nonlinear Elemental Stiffness Matrices}

The displacements are represented with linear shape functions, as in linear FEM. The calculation of the parameters $\alpha$, $\beta$, $\gamma$, $\delta$, $V$ (the volume of tetrahedron), and the shape functions are the same as it is done in linear FEM.

Nonlinear FEM differs from linear FEM because of the nonlinearity that arises from the higher order term neglected in Equation~\ref{eqn:3.5}. The strain vector that is used in linear FEM relies on the assumption that the displacements at x, y and z axes are very small. The initial and final positions of a given particle are practically the same; thus, the higher terms are neglected~\cite{BonetWood97}. When the displacements are large, however, this is no longer the case and one must distinguish between the initial and final coordinates of particles, so the higher order terms are added into the strain equations.

\begin{equation}
 (A'B')^2  = {\left(dx +  \frac{\partial{u_{x}}}{\partial{x}}dx \right)}^2 + {\left(\frac{\partial{u_{y}}}{\partial{x}}dx\right)}^2.
\label{eqn:4.1}
\end{equation}

\noindent By adding the higher order terms, 3D strains are defined as~\cite{Felippa96}:

\begin{equation}
\begin{array}{l}
\eta_{xx} = \frac{\partial{u_{x}}}{\partial{x}} +
\frac{1}{2}
\begin{bmatrix}
{\left(\frac{\partial{u_{x}}}{\partial{x}}\frac{\partial{u_{x}}}{\partial{x}}\right)}
+
{\left(\frac{\partial{u_{y}}}{\partial{x}}\frac{\partial{u_{y}}}{\partial{x}}\right)}
+
{\left(\frac{\partial{u_{z}}}{\partial{x}}\frac{\partial{u_{z}}}{\partial{x}}\right)}
\end{bmatrix} \\ \; \\
\eta_{yy} = \frac{\partial{u_{y}}}{\partial{y}} +
\frac{1}{2}
\begin{bmatrix}
{\left(\frac{\partial{u_{x}}}{\partial{y}}\frac{\partial{u_{x}}}{\partial{y}}\right)}
+
{\left(\frac{\partial{u_{y}}}{\partial{y}}\frac{\partial{u_{y}}}{\partial{y}}\right)}
+
{\left(\frac{\partial{u_{z}}}{\partial{y}}\frac{\partial{u_{z}}}{\partial{y}}\right)}
\end{bmatrix}\\ \;\\
\eta_{zz} = \frac{\partial{u_{z}}}{\partial{z}} +
\frac{1}{2}
\begin{bmatrix}
{\left(\frac{\partial{u_{x}}}{\partial{z}}\frac{\partial{u_{x}}}{\partial{z}}\right)}
+
{\left(\frac{\partial{u_{y}}}{\partial{z}}\frac{\partial{u_{y}}}{\partial{z}}\right)}
+
{\left(\frac{\partial{u_{z}}}{\partial{z}}\frac{\partial{u_{z}}}{\partial{z}}\right)}
\end{bmatrix}\\ \; \\
\gamma_{xy} = \frac{1}{2}{\left(\frac{\partial{u_{x}}}{\partial{y}} + \frac{\partial{u_{y}}}{\partial{x}}\right)}
+
\frac{1}{2}
\begin{bmatrix}
{\left(\frac{\partial{u_{x}}}{\partial{x}}\frac{\partial{u_{x}}}{\partial{y}}\right)}
+
{\left(\frac{\partial{u_{y}}}{\partial{x}}\frac{\partial{u_{y}}}{\partial{y}}\right)}
+
{\left(\frac{\partial{u_{z}}}{\partial{x}}\frac{\partial{u_{z}}}{\partial{y}}\right)}
\end{bmatrix}\\ \; \\
\gamma_{zx} = \frac{1}{2}{\left(\frac{\partial{u_{z}}}{\partial{x}} + \frac{\partial{u_{x}}}{\partial{z}}\right)}
+
\frac{1}{2}
\begin{bmatrix}
{\left(\frac{\partial{u_{x}}}{\partial{z}}\frac{\partial{u_{x}}}{\partial{x}}\right)}
+
{\left(\frac{\partial{u_{y}}}{\partial{z}}\frac{\partial{u_{y}}}{\partial{x}}\right)}
+
{\left(\frac{\partial{u_{z}}}{\partial{z}}\frac{\partial{u_{z}}}{\partial{x}}\right)}
\end{bmatrix}\\ \; \\
\gamma_{yz} = \frac{1}{2}{\left(\frac{\partial{u_{y}}}{\partial{z}} + \frac{\partial{u_{z}}}{\partial{y}}\right)}
+
\frac{1}{2}
\begin{bmatrix}
{\left(\frac{\partial{u_{x}}}{\partial{y}}\frac{\partial{u_{x}}}{\partial{z}}\right)}
+
{\left(\frac{\partial{u_{y}}}{\partial{y}}\frac{\partial{u_{y}}}{\partial{z}}\right)}
+
{\left(\frac{\partial{u_{z}}}{\partial{y}}\frac{\partial{u_{z}}}{\partial{z}}\right)}
\end{bmatrix}
\end{array}
\end{equation}

\noindent that leads to

\begin{equation}
\{\eta\} =
\begin{Bmatrix}
\eta_{xx} \\
\eta_{yy} \\
\eta_{zz} \\
2\left(\gamma_{xy} + \gamma_{yx}\right) \\
2\left(\gamma_{xz} + \gamma_{zx}\right) \\
2\left(\gamma_{yz} + \gamma_{zy}\right)
\end{Bmatrix}
=
\begin{Bmatrix}
\eta_{xx} \\
\eta_{yy} \\
\eta_{zz} \\
2\eta_{xy} \\
2\eta_{zx} \\
2\eta_{yz}
\end{Bmatrix}.
\label{eqn:4.2}
\end{equation}



\noindent Green-Lagrange strain tensor is represented in matrix notation as

\begin{equation}
\{\eta\} = [B_{L}] \{d\} + \frac{1}{2}\{d\}^T[B_{NL}]\{d\},
\label{eqn:4.3}
\end{equation}

\noindent where $\{d\}$ is the nodal displacements, [$B_{L}$] is the linear and $[B_{NL}]$ is the nonlinear part of the $[B_{0}]$ matrix~\cite{Pedersen06}. For a specific element, [$B_{L}$] and $[B_{NL}]$ are constant, as the $[B]$ matrix in linear FEM. With the modification of $\{d\}$ by introducing secant and tangent relations~\cite{Pedersen08}, Equation~\ref{eqn:4.3} becomes

\begin{equation}
\begin{array}{l}
\{\eta\} = \left([B_{L}] + \frac{1}{2}\{d^T\}[B_{NL}]\right)\{d\} = [B_{0}]\{d\}, \\
\{\bar{\eta}\} = \left([B_{L}] + \{d^T\}[B_{NL}]\right)\{d\} = [\bar{B_{0}}]\{d\}.
\end{array}
\label{eqn:4.3-1}
\end{equation}

The linear part of the $[B_{0}]$ matrix ($[B_{L}]$) is same as the $[B]$ matrix in linear FEM. The calculation of $[B_{0}]$ becomes more complex with the introduction of the nonlinear terms. After finding the nonlinear strains, these equations are combined with the shape functions to find matrix $[B_{0}]$

\begin{equation}
\{\bar{\eta}\} = [\bar{B_{0}}]\{d\}.
\label{eqn:4.3-2}
\end{equation}

\noindent The most frequently used terms for the calculation of the nonlinear strains are
$\frac{\partial{u_{x}}}{\partial{x}}$, $\frac{\partial{u_{x}}}{\partial{y}}$, $\frac{\partial{u_{x}}}{\partial{z}}$, $\frac{\partial{u_{y}}}{\partial{x}}$, $\frac{\partial{u_{y}}}{\partial{y}}$, $\frac{\partial{u_{y}}}{\partial{x}}$, $\frac{\partial{u_{z}}}{\partial{x}}$, $\frac{\partial{u_{z}}}{\partial{y}}$ and $\frac{\partial{u_{z}}}{\partial{z}}$. They are represented by

\begin{equation}
\begin{array}{l}
u_{xx} = (\beta_{1}u_{1} + \beta_{2}u_{2} + \beta_{3}u_{3} + \beta_{4}u_{4}) \\
u_{yx} = (\beta_{1}v_{1} + \beta_{2}v_{2} + \beta_{3}v_{3} + \beta_{4}v_{4}) \\
u_{zx} = (\beta_{1}w_{1} + \beta_{2}w_{2} + \beta_{3}w_{3} + \beta_{4}w_{4}) \\
u_{xy} = (\gamma_{1}u_{1} + \gamma_{2}u_{2} + \gamma_{3}u_{3} + \gamma_{4}u_{4}) \\
u_{yy} = (\gamma_{1}v_{1} + \gamma_{2}v_{2} + \gamma_{3}v_{3} + \gamma_{4}v_{4}) \\
u_{zy} = (\gamma_{1}w_{1} + \gamma_{2}w_{2} + \gamma_{3}w_{3} + \gamma_{4}w_{4}) \\
u_{xz} = (\delta_{1}u_{1} + \delta_{2}u_{2} + \delta_{3}u_{3} + \delta_{4}u_{4}) \\
u_{yz} = (\delta_{1}v_{1} + \delta_{2}v_{2} + \delta_{3}v_{3} + \delta_{4}v_{4}) \\
u_{zz} = (\delta_{1}w_{1} + \delta_{2}w_{2} + \delta_{3}w_{3} + \delta_{4}w_{4})
\end{array}
\label{eqn:4.3-3}
\end{equation}

\noindent where $u_{xx}$ represents $\frac{\partial{u_{x}}}{\partial{x}}$.

Using the linear parts of Equation~\ref{eqn:3.4} for displacements, we can evaluate the partial derivatives of the shape functions as follows (for the $1^{st}$ node of $[B_{NL}]$):

\begin{equation}
\begin{array}{lll}
\begin{bmatrix}
{\left(\frac{\partial{u_{x}}}{\partial{x}}\frac{\partial{u_{x}}}{\partial{x}}\right)}
+
{\left(\frac{\partial{u_{y}}}{\partial{x}}\frac{\partial{u_{y}}}{\partial{x}}\right)}
+
{\left(\frac{\partial{u_{z}}}{\partial{x}}\frac{\partial{u_{z}}}{\partial{x}}\right)}
\end{bmatrix}
& = &
\frac{1}{6V}(\beta_{1} (u_{xx} + u_{yx} + u_{zx})) \\ \; \\
\begin{bmatrix}
{\left(\frac{\partial{u_{x}}}{\partial{y}}\frac{\partial{u_{x}}}{\partial{y}}\right)}
+
{\left(\frac{\partial{u_{y}}}{\partial{y}}\frac{\partial{u_{y}}}{\partial{y}}\right)}
+
{\left(\frac{\partial{u_{z}}}{\partial{y}}\frac{\partial{u_{z}}}{\partial{y}}\right)}
\end{bmatrix}
& = &
\frac{1}{6V}(\gamma_{1} (u_{xy} + u_{yy} + u_{zy})) \\ \; \\
\begin{bmatrix}
{\left(\frac{\partial{u_{x}}}{\partial{z}}\frac{\partial{u_{x}}}{\partial{z}}\right)}
+
{\left(\frac{\partial{u_{y}}}{\partial{z}}\frac{\partial{u_{y}}}{\partial{z}}\right)}
+
{\left(\frac{\partial{u_{z}}}{\partial{z}}\frac{\partial{u_{z}}}{\partial{z}}\right)}
\end{bmatrix}
& = &
\frac{1}{6V}(\delta_{1} (u_{xz} + u_{yz} + u_{zz}))  \; \\ \\
\begin{bmatrix}
{\left(\frac{\partial{u_{x}}}{\partial{x}}\frac{\partial{u_{x}}}{\partial{y}}\right)}
+
{\left(\frac{\partial{u_{y}}}{\partial{x}}\frac{\partial{u_{y}}}{\partial{y}}\right)}
+
{\left(\frac{\partial{u_{z}}}{\partial{x}}\frac{\partial{u_{z}}}{\partial{y}}\right)}
\end{bmatrix}
& = &   \frac{1}{6V}(\gamma_{1} (u_{xx} + u_{yx} + u_{zx})) + \\
&   &   \frac{1}{6V}(\beta_{1} (u_{xy} + u_{yy} + u_{zy}))  \\ \; \\
\begin{bmatrix}
{\left(\frac{\partial{u_{x}}}{\partial{z}}\frac{\partial{u_{x}}}{\partial{x}}\right)}
+
{\left(\frac{\partial{u_{y}}}{\partial{z}}\frac{\partial{u_{y}}}{\partial{x}}\right)}
+
{\left(\frac{\partial{u_{z}}}{\partial{z}}\frac{\partial{u_{z}}}{\partial{x}}\right)}
\end{bmatrix}
& = &  \frac{1}{6V}(\delta_{1} (u_{xx} + u_{yx} + u_{zx})) + \\
&   &  \frac{1}{6V}(\beta_{1} (u_{xz} + u_{yz} + u_{zz}))  \\ \; \\
\begin{bmatrix}
{\left(\frac{\partial{u_{x}}}{\partial{y}}\frac{\partial{u_{x}}}{\partial{z}}\right)}
+
{\left(\frac{\partial{u_{y}}}{\partial{y}}\frac{\partial{u_{y}}}{\partial{z}}\right)}
+
{\left(\frac{\partial{u_{z}}}{\partial{y}}\frac{\partial{u_{z}}}{\partial{z}}\right)}
\end{bmatrix}
& = &   \frac{1}{6V}(\gamma_{1} (u_{xz} + u_{yz} + u_{zz})) + \\
&   &   \frac{1}{6V}(\delta_{1} (u_{xy} + u_{yy} + u_{zy}))
\end{array}
\label{eqn:4.3-5}
\end{equation}


\noindent Using Equations~\ref{eqn:4.3-1}~and~\ref{eqn:4.3-5}, we obtain $[\bar{B_{0}}]$ for the 1$^{st}$ node

\begin{equation}
[\bar{B}_{0_1}]
 =
\begin{bmatrix}
\beta_{1} + \beta_{1}(u_{xx})  & \beta_{1}(u_{yx}) & \beta_{1}(u_{zx}) \\
\gamma_{1}(u_{xy}) & \gamma_{1} + \gamma_{1}(u_{yy}) & \gamma_{1}(u_{zy}) \\
\delta_{1}(u_{xz}) & \delta_{1}(u_{yz}) & \delta_{1} + \delta_{1}(u_{zz}) \\
\gamma_{1} + \gamma_{1}(u_{xx}) + \beta_{1}(u_{xy}) & \gamma_{1}(u_{yx}) + \beta_{1} + \beta_{1}(u_{yy}) & \gamma_{1}(u_{zx}) + \beta_{1}(u_{zy}) \\
\delta_{1} + \delta_{1}(u_{xx}) + \beta_{1}(u_{xz}) & \delta_{1}(u_{yx}) + \beta_{1}(u_{yz})  & \delta_{1}(u_{zx}) + \beta_{1} + \beta_{1}(u_{zz}) \\
\gamma_{1}(u_{xz}) + \delta_{1}(u_{xy}) & \gamma_{1} + \gamma_{1}(u_{yz}) + \delta_{1}(u_{yy}) & \gamma_{1}(u_{zz}) + \delta_{1} + \delta_{1}(u_{zy})
\end{bmatrix}
\begin{Bmatrix}
u_{1} \\
v_{1} \\
w_{1}
\end{Bmatrix}.
 \label{eqn:4.3-6}
\end{equation}

\noindent Similarly, using Equations~\ref{eqn:4.3-1}~and~\ref{eqn:4.3-5}, we obtain $[B_{0}]$ for the 1$^{st}$ node

\begin{equation}
\begin{array}{l}
[B_{0_1}]
 = \\
\begin{bmatrix}
\beta_{1} + \frac{1}{2}\beta_{1}(u_{xx})  & \frac{1}{2}\beta_{1}(u_{yx}) & \frac{1}{2}\beta_{1}(u_{zx}) \\
\frac{1}{2}\gamma_{1}(u_{xy}) & \gamma_{1} + \frac{1}{2}\gamma_{1}(u_{yy}) & \frac{1}{2}\gamma_{1}(u_{zy}) \\
\frac{1}{2}\delta_{1}(u_{xz}) & \frac{1}{2}\delta_{1}(u_{yz}) & \delta_{1} + \frac{1}{2}\delta_{1}(u_{zz}) \\
\gamma_{1} + \frac{1}{2}\gamma_{1}(u_{xx}) + \frac{1}{2}\beta_{1}(u_{xy}) & \frac{1}{2}\gamma_{1}(u_{yx}) + \beta_{1} + \frac{1}{2}\beta_{1}(u_{yy}) & \frac{1}{2}\gamma_{1}(u_{zx}) + \frac{1}{2}\beta_{1}(u_{zy}) \\
\delta_{1} + \frac{1}{2}\delta_{1}(u_{xx}) + \frac{1}{2}\beta_{1}(u_{xz}) & \frac{1}{2}\delta_{1}(u_{yx}) + \frac{1}{2}\beta_{1}(u_{yz})  & \frac{1}{2}\delta_{1}(u_{zx}) + \beta_{1} + \frac{1}{2}\beta_{1}(u_{zz}) \\
\frac{1}{2}\gamma_{1}(u_{xz}) + \frac{1}{2}\delta_{1}(u_{xy}) & \gamma_{1} + \frac{1}{2}\gamma_{1}(u_{yz}) + \frac{1}{2}\delta_{1}(u_{yy}) & \frac{1}{2}\gamma_{1}(u_{zz}) + \delta_{1} + \frac{1}{2}\delta_{1}(u_{zy})
\end{bmatrix}
\begin{Bmatrix}
u_{1} \\
v_{1} \\
w_{1}
\end{Bmatrix}
\end{array}.
\label{eqn:4.3-7}
\end{equation}

\noindent From Equation~\ref{eqn:1.1}, the engineering stress vector $\tau$ is related to the strain vector by

\begin{equation}
\tau = [E]\{\bar{\eta}\} = [E][\bar{B_{0}}]\{d\}.
\label{eqn:4.4}
\end{equation}

\noindent From the conservation of the potential energy, substituting Equations~\ref{eqn:4.3}~and~\ref{eqn:4.4} into Equation~\ref{eqn:1.1}, we obtain the element stiffness matrix

\begin{equation}
[k(u)] = \int\int\int\{d\}^T[B_{0}]^T[E][\bar{B_{0}}]\{d\} dx \; dy \; dz.
\label{eqn:4.5}
\end{equation}

\noindent We can discard the integrals as we did for linear FEM. $[B_{0}]$, $[E]$ and $[\bar{B_{0}}]$ are constant for tetrahedral element, so that Equation~\ref{eqn:4.5} is rewritten by
\begin{equation}
[k(u)] = \{d\}^T[B_{0}]^T[E][\bar{B_{0}}]\{d\} V.
\label{eqn:4.5-1}
\end{equation}

\noindent Introducing nodal forces, we obtain

\begin{equation}
\begin{Bmatrix}
 f
 \end{Bmatrix}
 =
\begin{Bmatrix}
f_{1x} \\ f_{1y} \\ f_{1z} \\ \vdots \\ f_{4x} \\ f_{4y} \\ f_{4z}
\end{Bmatrix}
\{d\}^T.
\label{eqn:4.5-2}
\end{equation}

\noindent With the equilibrium equation, and the cancellation of the $\{d\}^T$ the whole system for one element reduces to

\begin{equation}
k(d)^e\{d\}^e = f^e.
\label{eqn:4.6}
\end{equation}

\noindent By substituting $\{d\}$ by $u$, we obtain

\begin{equation}
k(u)^{e}u^{e} = f^{e}.
\label{eqn:4.6-1}
\end{equation}

\noindent Finally, there are only nonlinear displacement functions left, which are solved with the Newton-Raphson method to find the unknown displacements ${u}$.

\subsection{Construction of Nonlinear Element Residuals}

Element residuals are necessary for the iterative Newton-Raphson method. The element residual is a $12 \times 1$ vector for a specific element. The residual for a specific element is defined as

\begin{equation}
r^{e} = k(u)^{e} - f^{e}.
\label{eqn:4.6-2}
\end{equation}

\noindent Having determined $r^{e}$, we can now express Equation~\ref{eqn:4.6-2} in expanded vector form as

\begin{equation}
\begin{Bmatrix}
r_{1} \\
r_{2} \\
r_{3} \\
\vdots \\
r_{12}
\end{Bmatrix}
=
\begin{bmatrix}
k(u)_{(1,1)} + k(u)_{(1,2)} + k(u)_{(1,3)} + \hdots + k(u)_{(1,12)} \\
k(u)_{(2,1)} + k(u)_{(2,2)} + k(u)_{(2,3)} + \hdots + k(u)_{(2,12)} \\
k(u)_{(3,1)} + k(u)_{(3,2)} + k(u)_{(3,3)} + \hdots + k(u)_{(3,12)} \\
\vdots \\
k(u)_{(12,1)} + k(u)_{(12,2)} + k(u)_{(12,3)} + \hdots + k(u)_{(12,12)}
\end{bmatrix}
-
\begin{Bmatrix}
f_{1} \\ f_{2} \\ f_{3} \\ \vdots \\  f_{12}
\end{Bmatrix}.
\label{eqn:4.6-3}
\end{equation}

The tangent stiffness matrix $[K]_{T}^e$ ($r'^{e}$) is also necessary for the iterative Newton-Raphson method. The tangent stiffness matrix is also $12 \times 12$ matrix, like the elemental stiffness matrix. However, the tangent stiffness matrix depends on residuals, unlike the elemental stiffness matrix. Elemental stiffness matrices are used to construct residuals and the derivatives of the residuals are used to construct the elemental tangent stiffness matrices. We can express the elemental tangent stiffness matrix for a specific element as

\begin{equation}
r'^{e}
=
[K]_{T}^e
=
\begin{bmatrix}
\frac{\partial{}}{\partial{u_{1}}}r_{1} & \frac{\partial{}}{\partial{v_{1}}}r_{1} & \frac{\partial{}}{\partial{w_{1}}}r_{1} & \hdots & \frac{\partial{}}{\partial{u_{4}}}r_{1} & \frac{\partial{}}{\partial{v_{4}}}r_{1} & \frac{\partial{}}{\partial{w_{4}}}r_{1} \\ \vspace*{1.0ex}
\frac{\partial{}}{\partial{u_{1}}}r_{2} & \frac{\partial{}}{\partial{v_{1}}}r_{2} & \frac{\partial{}}{\partial{w_{1}}}r_{2} & \hdots & \frac{\partial{}}{\partial{u_{4}}}r_{2} & \frac{\partial{}}{\partial{v_{4}}}r_{2} & \frac{\partial{}}{\partial{w_{4}}}r_{2} \\ \vspace*{1.0ex}
\frac{\partial{}}{\partial{u_{1}}}r_{3} & \frac{\partial{}}{\partial{v_{1}}}r_{3} & \frac{\partial{}}{\partial{w_{1}}}r_{3} & \hdots & \frac{\partial{}}{\partial{u_{4}}}r_{3} & \frac{\partial{}}{\partial{v_{4}}}r_{3} & \frac{\partial{}}{\partial{w_{4}}}r_{3} \\ \vspace*{1.0ex}
\vdots & \vdots & \vdots & \vdots & \vdots & \vdots & \vdots \\ \vspace*{1.0ex}
\frac{\partial{}}{\partial{u_{1}}}r_{12} & \frac{\partial{}}{\partial{v_{1}}}r_{12} & \frac{\partial{}}{\partial{w_{1}}}r_{12} & \hdots & \frac{\partial{}}{\partial{u_{4}}}r_{12} & \frac{\partial{}}{\partial{v_{4}}}r_{12} & \frac{\partial{}}{\partial{w_{4}}}r_{12} \vspace*{1.0ex}
\end{bmatrix}.
\label{eqn:4.6-4}
\end{equation}


\subsection{Solution of the Non-linear System with Newton-Raphson Method}

Newton-Raphson method is a fast and popular numerical method for solving nonlinear equations~\cite{Kelly87}, as compared to the other methods, such as direct iteration. In principle, the method works by applying two steps (cf. Algorithm~\ref{alg:newton-raphson}): (i)~check if the equilibrium is reached within the desired accuracy; (ii)~if not, make a suitable adjustment to the state of the deformation~\cite{Krenk09}. An initial guess for displacements are needed to start the iterations. The displacements are updated according to

\begin{equation}
x_{k+1} = x_{k} - \frac{f_{x_{k}}}{f'_{x_{k}}}.
\label{eqn:4.7-1}
\end{equation}

\begin{algorithm}
\caption{Newton-Raphson method}
\label{alg:newton-raphson}
{
\fontsize{10}{10}\selectfont
\begin{algorithmic}[l]
\STATE Make initial guess $f(x)$          			\vspace*{1.0ex}
\WHILE{ $|f(x)| \le \delta$ }             			\vspace*{1.0ex}
     \STATE Compute  $p = -\frac{f(x)}{f'(x)}$      \vspace*{1.0ex}
     \STATE Update $x = x + p$            			\vspace*{1.0ex}
     \STATE Calculate $f(x)$              			\vspace*{1.0ex}
\ENDWHILE

\end{algorithmic}
}
\end{algorithm}

In our nonlinear solution, $u$ is the vector that keeps the information of the nodal displacements. Instead of making only one assumption, we make whole $u$ vector initial guess in order to start the iteration.

\begin{equation}
u_{1} = u_{0} - \frac{r_{u_{0}}}{r'_{u_{0}}},
\label{eqn:4.7}
\end{equation}

\noindent where $r$ is residual of the global stiffness matrix $[K]$ calculated in Equation~\ref{eqn:4.6-3} and $r'$ is the tangent stiffness matrix calculated in Equation~\ref{eqn:4.6-4}.

At every step, the vector $r$ and the matrix $r'$ are updated for every element with the new $u_{i}$ values. Then, $r$ and $r'$ are assembled as we did with for the global stiffness matrix $K$ and the global force vector $F$ in linear FEM. Boundary conditions are applied to the global $r$ vector and the global $r'$ matrix. Using the global $r$ vector and the global $r'$ matrix, we have

\begin{equation}
\begin{array}{l}
{r'(u_{i})}p = -{r(u_{i})},\;\; \text{and}\\
p = -(r'(u_{i}))^{-1}r(u_{i}).
\end{array}
\label{eqn:4.9}
\end{equation}

\noindent $u_{i}$ is updated with the solution of Equation~\ref{eqn:4.9}.

\begin{equation}
u_{i+1} = u_{i} + p.
\label{eqn:4.10}
\end{equation}

\noindent Then, we check if the equilibrium is reached within the desired accuracy defined by $\delta$ as

\begin{equation}
|r(u_{i})| \le \delta.
\label{eqn:4.11}
\end{equation}

\noindent After the desired accuracy is reached, the unknown nodal displacements are found.

\section{Verification of the Proposed Approach}

Verification is one of the important steps of the finite element analysis. We verified our approach with Pedersen's analytical stiffness matrices for tetrahedral elements solution~\cite{Pedersen06}. In the experiments we obtained the same displacement amount with his method. In this section, Pedersen's method is explained in order to see the differences between our approach and his approach.

Both approaches give the same results since they use the same Green-Lagrange strains and tetrahedral elements. However, the computation times differ because of different methods to calculate the stiffness matrices. Pedersen divides the elemental stiffness matrices $S$ into nine sub-matrices, $[S_{xx}]$, $[S_{xy}]$, $[S_{xz}]$, $[S_{yx}]$, $[S_{yy}]$, $[S_{yz}]$, $[S_{zx}]$, $[S_{zy}]$ and $[S_{zz}]$, which is represented as $K$ ($12 \times 12$ stiffness matrix) in our method.

\begin{equation}
\begin{array}{l}
S[1:4,1:4] = [S_{xx}] \\
S[1:4,5:8] = [S_{xy}] \\
S[1:4,9:12] = [S_{xz}] \\
S[5:8,1:4] = [S_{yx}] \\
S[5:8,5:8] = [S_{yy}] \\
S[5:8,9:12] = [S_{yz}] \\
S[9:12,1:4] = [S_{zx}] \\
S[9:12,5:8] = [S_{zy}] \\
S[9:12,9:12] = [S_{zz}]
\end{array}
\label{eqn:4.12}
\end{equation}

\noindent These nine sub-matrices are calculated with 81 linear combination factors. Pedersen obtains $[S]_{xx}$ as

\begin{eqnarray}
[S]_{xx}
& = &
A_{xxxx}[T_{xx}] + A_{xxyy}[T_{yy}] + A_{xxzz}[T_{zz}] + A_{xxxy}[T_{xy}] + A_{xxyx}[T_{xy}^{T}] + \nonumber \\
&  & A_{xxxz}[T_{xz}] + A_{xxzx}[T_{xz}^{T}] + A_{xxyz}[T_{yz}] + A_{xxzy}[T_{yz}^{T}]
\label{eqn:4.13}
\end{eqnarray}

\noindent $[T]$ sub-matrices coincide with the linear part of our global stiffness matrix $[K]$. In other words, the matrix $[T_{xx}]$ is obtained by

\begin{equation}
\begin{array}{l}
[T_{xx}]
=
\begin{bmatrix}
q_{x}^2 & -p_{5968}q_{x} & -p_{3829}q_{x} & -p_{2635}q_{x} \\
-p_{5968}q_{x} & p_{5968}^2 & p_{5968}p_{3829} & p_{5968}p_{2635} \\
-p_{3829}q_{x} & p_{5968}p_{3829} & p_{23829} & p_{3829}p_{2635} \\
-p_{2635}q_{x} & p_{5968}p_{2635} & p_{3829}p_{2635} & p_{2635}^2
\end{bmatrix}
\end{array}
\label{eqn:4.14}
\end{equation}

\noindent following the short notation defined by Pedersen, e.g., $p_{5968} = p_{5}p_{9} - p_{6}p_{8}$.
When we expand the unknown term, $q_{x}$, in Equation~\ref{eqn:4.14}, it becomes $-\beta_{1}$  in our method:

 \begin{equation}
\begin{array}{l}
q_{x} = p_{5968} + p_{3829} + p_{2635} \\
q_{x} = (y_{3}z_{4} - y_{4}z_{3}) + (z_{2}y_{4} - z_{4}y_{2}) + (y_{2}z_{3} - y_{3}z_{2}) \\
\beta_{1} =  -y_{3}z_{4} + y_{4}z_{3} - z_{2}y_{4} + z_{4}y_{2} - y_{2}z_{3} + y_{3}z_{2} \\
q_{x} = -\beta_{1}
\end{array}
\label{eqn:4.15}
\end{equation}

\noindent As it is seen from Equation~\ref{eqn:4.16}, the other terms that Pedersen used are the same as the ones used in our method.

\begin{equation}
\begin{array}{l}
q_{x} = -\beta_{1} \\
p_{5968} = \beta_{2} \\
p_{3829} = \beta_{3} \\
p_{2635} = \beta_{4} \\
q_{y} = -\gamma_{1} \\
p_{6749} = \gamma_{2} \\
p_{1937} = \gamma_{3} \\
p_{3416} = \gamma_{4} \\
q_{z} = \delta_{1} \\
p_{4857} = \delta_{2} \\
p_{2718} = \delta_{3} \\
p_{1524}= \delta_{4}
\end{array}
\label{eqn:4.16}
\end{equation}

Apart from the stiffness matrix calculation, the solutions of the nonlinear equations in both methods are the same. Both approaches use the Newton-Raphson method to find the unknown displacements. Hence, the comparison of the computation time required to calculate the stiffness matrices is sufficient to compare the performances of two approaches.


