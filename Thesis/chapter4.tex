\chapter{3-D Model}
\label{chapter4}

There are two types of 3-D models I needed for my thesis project: A cloth mesh to be simulated and a human body to dress with the simulated cloth.

\section{Human Avatar}
\label{section4_1}

Although I have experience with modeling in Blender, designing a 3-D human avatar from scratch seemed unnecessary, as it is not in my area of expertise and there are available models online. After a vast search, I decided to utilize the model I have acquired from a forum, as it was realistic in appearance, proportions and details \cite{Mmava2012}. In order to make it look more realistic, I needed to implement additional features.

\subsection{Rigging}

Animating a 3D mesh requires skinning, which is moving the vertices with respect to a bone on a skeleton. The model I acquired from internet had no skinning or material information, which I meant I had to do it myself. And even it was a pre-skinned model, I would prefer doing it on my own, in order to be able to integrate the model easily into my software. 

I performed the skinning in blender, where I used the predefined human skeleton, detailed in H-ANIM2 level with a much simplified spine (Figure \ref{fig:rigging_skeleton}). 

\begin{figure}[h]
\centerline{\fbox{\psfig{figure=figures/rigging_skeleton.png,width=1.00\textwidth}}}
\caption{The Rigging Base Skeleton}
\label{fig:rigging_skeleton}
\end{figure}

Prior to rigging, the skeleton needed to match the given mesh in position. After modifying the initial bone positions, the vertex groups are assigned to the bones with proper weights (Figure \ref{fig:weight_humerus_r}). After assigning every vertex to the appropriate bones, there were no cracked surfaces with motions which are natural for humans. 

In the end, the model was exported in OGRE .mesh format along with the skeleton it used.

\begin{figure}[h]
\centerline{\fbox{\psfig{figure=figures/weight_humerus_r.png,width=1.00\textwidth}}}
\caption{The vertex weights for the Humerus.R bone}
\label{fig:weight_humerus_r}
\end{figure}

\subsection{Materials}

The base model came with no material information. In order to improve the realism of the model, I implemented texturing/painting for various parts of the body. These parts include the general skin, eyes, nails and lips so far (Figure \ref{fig:detailed_face}).  I will continue material introduction to other parts I deem necessary to improve realism. I also added non-simulated hair and some accessories such as hair sticks and earrings to the model to make it look more realistic.

\begin{figure}[h]
\centerline{\fbox{\psfig{figure=figures/detailed_face.png,width=1.00\textwidth}}}
\caption{Detailed Materials on the Face}
\label{fig:detailed_face}
\end{figure}

\section{Cloth Mesh}
\label{section4_2}

A major part of my thesis is on accurate cloth simulation. Next to the required software and algorithms, I also needed accurately modeled cloth meshes to be simulated. After my searching and filtering through a variety of models online, I decided to work with a dress model \cite{LadyJewell2012}. The model comes as pure mesh, without any material information, although it was detailed well and sufficient for my research.

\subsection{Body Positioning and Splitting the Dress Mesh}

To correctly animate and simulate the dress on a human avatar, I needed them to be in proportion with each other and properly aligned. The initial positions and proportions of the human avatar and the dress mesh were set in Blender, as it was easier to perform this with a visual tool (Figure \ref{fig:dress_and_body}).
After various attempts to simulate all the vertices on the dress mesh, I failed to achieve a realistic looking results due to two main reasons:
\begin{itemize}
  \item The whole dress consisted of 4088 vertices. Although I was able to simulate in real-time, too many vertices resulted in the simulation algorithm to break down due to the very large number of vertices in the cloth. It shifted from a fabric structure to more of a jelly form, over stretching and tearing with its own weight.
  \item The friction necessary to keep the dress on the Human Avatars shoulders was not sufficient enough to keep the dress on the avatar. It kept sliding, stretching and acting unnaturally.
\end{itemize}

In order to overcome these problems, I decided to split the dress mesh into two parts, and utilize different animation techniques on them.

\begin{itemize}
  \item The top part, which should be attached to the body, not blowing in the wind, was modeled as a static mesh, with skinning like the human avatar. It was animated with the same transformations as the human avatar, matching its position and staying on the body perfectly.
  \item The bottom part was the part to be simulated, affected by inertia, wind and other factors. Their attachment line is just above the waist of the human avatar, which was confirmed after various experimentations with other locations (Figure \ref{fig:dress_and_body}). Keeping the line too below resulted in both unnatural collisions and the cloth being too rigid. Keeping it too high brought out the original problems. With its current setting, the dress mesh is animated naturally on the virtual avatar.
\end{itemize}

\begin{figure}[h]
\centerline{\fbox{\psfig{figure=figures/dress_and_body.png,width=1.00\textwidth}}}
\caption{The  Dress, positioned on the body, along with the upper-part skeleton. In this shot, the bottom part is highlighted in orange border.}
\label{fig:dress_and_body}
\end{figure}

