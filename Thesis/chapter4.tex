\chapter{Linear Finite Element Method}
\label{chapter4}

This chapter describes in detail the linear FEM method, development of small deformation strains that lead to linear FEM, the stiffness matrix, and the solution of the linear FEM.

\section{Linear FEM Using Tetrahedral Elements}
\label{sec3LinFEM}

Most of the linear FEM methods for 3D tetrahedral elements consist of five stages:

\begin{enumerate}
\item Tetrahedralization;
\item Construction of elemental stiffness matrices;
\item Assembly of elemental stiffness matrices and force vectors;
\item Applying boundary conditions;
\item Solving the linear system that gives unknown nodal displacements.
\end{enumerate}


\subsection{Tetrahedralization}

We use same tetrahedralized meshes for both linear and non-linear solutions to provide integrity and make a better analysis for our experiments. We use the Application Programming Interface of TetGen~\cite{Tetgen2011} as a tetrahedral mesh generator and integrated it into our implementation. The tetrahedralization process produces the nodal positions and the elements for each node. We use nodal positions to construct elemental stiffness matrices, and element's nodal information to assemble the elemental stiffness matrices to form the global stiffness matrix.

\subsection{Construction of Elemental Stiffness Matrices}

\begin{figure}[h]
\centerline{\fbox{\psfig{figure=figures/tetElement.ps,width=0.60\textwidth}}}
\caption{Tetrahedral element}
\label{fig:tetelement}
\end{figure}

We use tetrahedral elements for modeling meshes in the experiments (Figure~\ref{fig:tetelement}). Overall, there are 12 unknown nodal displacements in a tetrahedral element. They are given by~\cite{Logan07}

\begin{equation}
\begin{Bmatrix}
 d
 \end{Bmatrix}
 =
 \begin{Bmatrix}
 u(x,y,z)
 \end{Bmatrix}
 =
\begin{Bmatrix} u_{1} \\ v_{1} \\ w_{1} \\ \vdots \\ u_{4} \\ v_{4} \\ w_{4} \end{Bmatrix}.
\label{eqn:3.3}
\end{equation}

\noindent In global coordinates, we represent displacements by linear function by

\begin{equation}
u^e(x,y,z) = c_{1} + c_{2}x + c_{3}y + c_{4}z.
\label{eqn:3.01}
\end{equation}

\noindent For all 4 vertices, Equation~\ref{eqn:3.01} is extended as

\begin{equation}
\begin{bmatrix}
1 & x_{1} & y_{1} & z_{1} \\
1 & x_{2} & y_{2} & z_{2} \\
1 & x_{3} & y_{3} & z_{3} \\
1 & x_{4} & y_{4} & z_{4}
\end{bmatrix}
\begin{Bmatrix}
c_{1} \\ c_{2} \\ c_{3}  \\ c_{4}
\end{Bmatrix}
=
\begin{Bmatrix}
u_{1} \\ u_{2} \\ u_{3}  \\ u_{4}
\end{Bmatrix}.
\label{eqn:3.02}
\end{equation}

\noindent Constants $c_{n}$ can be found as

\begin{equation}
c_{n} = v^{-1} u_{n},
\label{eqn:3.03}
\end{equation}

\noindent where $v^{-1}$ is given by

\begin{equation}
v^{-1} =
\frac{1}{det(v)}
\begin{bmatrix}
\alpha_{1} & \alpha_{2} & \alpha_{3} & \alpha_{4} \\
\beta_{1} & \beta_{2} & \beta_{3} & \beta_{4} \\
\gamma_{1} & \gamma_{2} & \gamma_{3} & \gamma_{4} \\
\delta_{1} & \delta_{2} & \delta_{3} & \delta_{4}
\end{bmatrix}.
\label{eqn:3.04}
\end{equation}

\noindent $det(v)$ is $6V$, where $V$ is the volume of the tetrahedron. If we substitute Equation~\ref{eqn:3.03} into Equation~\ref{eqn:3.01}, we obtain

\begin{equation}
u^e(x,y,z)
=
\frac{1}{6V^{e}}
\begin{bmatrix}
1 & x & y & z
\end{bmatrix}
\begin{bmatrix}
\alpha_{1} & \alpha_{2} & \alpha_{3} & \alpha_{4} \\
\beta_{1} & \beta_{2} & \beta_{3} & \beta_{4} \\
\gamma_{1} & \gamma_{2} & \gamma_{3} & \gamma_{4} \\
\delta_{1} & \delta_{2} & \delta_{3} & \delta_{4}
\end{bmatrix}
\begin{Bmatrix}
u_{1} \\ u_{2} \\ u_{3}  \\ u_{4}
\end{Bmatrix}.
\label{eqn:3.05}
\end{equation}

\noindent $\alpha$, $\beta$, $\gamma$, $\delta$ and the volume $V$ are calculated by

\begin{equation}
{\alpha_{1}
=
\begin{vmatrix}
x_{2} & y_{2} & z_{2} \\
x_{3} & y_{3} & z_{3} \\
x_{4} & y_{4} & z_{4}
\end{vmatrix}},\;\;
{\alpha_{2}
= -
\begin{vmatrix}
x_{1} & y_{1} & z_{1} \\
x_{3} & y_{3} & z_{3} \\
x_{4} & y_{4} & z_{4}
\end{vmatrix}},\;\;
{\alpha_{3}
=
\begin{vmatrix}
x_{1} & y_{1} & z_{1} \\
x_{2} & y_{2} & z_{2} \\
x_{4} & y_{4} & z_{4}
\end{vmatrix}},\;\;
{\alpha_{4}
= -
\begin{vmatrix}
x_{1} & y_{1} & z_{1} \\
x_{2} & y_{2} & z_{2} \\
x_{3} & y_{3} & z_{3}
\end{vmatrix}}
\label{fig:alphacalc}
\end{equation}


\begin{equation}
{\beta_{1}
= -
\begin{vmatrix}
1 & y_{2} & z_{2} \\
1 & y_{3} & z_{3} \\
1 & y_{4} & z_{4}
\end{vmatrix}},\;\;
{\beta_{2}
=
\begin{vmatrix}
1 & y_{1} & z_{1} \\
1 & y_{3} & z_{3} \\
1 & y_{4} & z_{4}
\end{vmatrix}},\;\;
{\beta_{3}
= -
\begin{vmatrix}
1 & y_{1} & z_{1} \\
1 & y_{2} & z_{2} \\
1 & y_{4} & z_{4}
\end{vmatrix}},\;\;
{\beta_{4}
=
\begin{vmatrix}
1 & y_{1} & z_{1} \\
1 & y_{2} & z_{2} \\
1 & y_{3} & z_{3}
\end{vmatrix}}
\label{fig:betacalc}
\end{equation}

\begin{equation}
{\gamma_{1}
=
\begin{vmatrix}
1 & x_{2} & z_{2} \\
1 & x_{3} & z_{3} \\
1 & x_{4} & z_{4}
\end{vmatrix}},\;\;
{\gamma_{2}
= -
\begin{vmatrix}
1 & x_{1} & z_{1} \\
1 & x_{3} & z_{3} \\
1 & x_{4} & z_{4}
\end{vmatrix}},\;\;
{\gamma_{3}
=
\begin{vmatrix}
1 & x_{1} & z_{1} \\
1 & x_{2} & z_{2} \\
1 & x_{4} & z_{4}
\end{vmatrix}},\;\;
{\gamma_{4}
= -
\begin{vmatrix}
1 & x_{1} & z_{1} \\
1 & x_{2} & z_{2} \\
1 & x_{3} & z_{3}
\end{vmatrix}}
\label{fig:gammacalc}
\end{equation}

\begin{equation}
{\delta_{1}
= -
\begin{vmatrix}
1 & x_{2} & y_{2} \\
1 & x_{3} & y_{3} \\
1 & x_{4} & y_{4}
\end{vmatrix}},\;\;
{\delta_{2}
=
\begin{vmatrix}
1 & x_{1} & y_{1} \\
1 & x_{3} & y_{3} \\
1 & x_{4} & y_{4}
\end{vmatrix}},\;\;
{\delta_{3}
= -
\begin{vmatrix}
1 & x_{1} & y_{1} \\
1 & x_{2} & y_{2} \\
1 & x_{4} & y_{4}
\end{vmatrix}},\;\;
{\delta_{4}
=
\begin{vmatrix}
1 & x_{1} & y_{1} \\
1 & x_{2} & y_{2} \\
1 & x_{3} & y_{3}
\end{vmatrix}}
\label{fig:deltacalc}
\end{equation}

\begin{equation}
{6V
=
\begin{vmatrix}
1 & x_{i} & y_{i} & z_{i} \\
1 & x_{j} & y_{j} & z_{j} \\
1 & x_{k} & y_{k} & z_{k} \\
1 & x_{l} & y_{l} & z_{l}
\end{vmatrix}}
\label{eqn:volcalc}
\end{equation}

\noindent Because of the differentials in strain calculation, $\alpha$ is not used in the following stages.
If we expand Equation~\ref{eqn:3.05}, we obtain



\begin{equation}
\begin{array}{l}
u^{e}(x,y,z) = \\
\frac{1}{6V^{e}}
\begin{bmatrix}
\alpha_{1} + \beta_{1}x + \gamma_{1}y + \delta_{1}z \\
\alpha_{2} + \beta_{2}x + \gamma_{2}y + \delta_{2}z \\
\alpha_{3} + \beta_{3}x + \gamma_{3}y + \delta_{3}z \\
\alpha_{4} + \beta_{4}x + \gamma_{4}y + \delta_{4}z
\end{bmatrix}
\begin{bmatrix}
u(x,y,z)_{1} & u(x,y,z)_{2} & u(x,y,z)_{3} & u(x,y,z)_{4}
\end{bmatrix}
\end{array}
\label{eqn:3.06}
\end{equation}

\noindent For tetrahedral elements, to express displacements in simpler form, shape functions are introduced (${\psi}_{1}, {\psi}_{2}, {\psi}_{3}, {\psi}_{4}$). They are given by

\begin{equation}
\begin{array}{l}
{\psi}_{1} =  \frac{1}{6V}\left(\alpha_{1} + \beta_{1}x + \gamma_{1}y + \delta_{1}z\right)u(x,y,z)_{1} \\
{\psi}_{2} =  \frac{1}{6V}\left(\alpha_{2} + \beta_{2}x + \gamma_{2}y + \delta_{2}z\right)u(x,y,z)_{2} \\
{\psi}_{3} =  \frac{1}{6V}\left(\alpha_{3} + \beta_{3}x + \gamma_{3}y + \delta_{3}z\right)u(x,y,z)_{3} \\
{\psi}_{4} =  \frac{1}{6V}\left(\alpha_{4} + \beta_{4}x + \gamma_{4}y + \delta_{4}z\right)u(x,y,z)_{4}
\end{array}
\label{eqn:3.4}
\end{equation}

The next step is to find the infinitesimal strains that are used to calculate the global stiffness matrix. Figure~\ref{fig:deformation2d} shows that the element edge that lies on x-axis $AB$ becomes $A'B'$. The engineering normal strain is calculated as the change in the length of the line~\cite{Logan07}:

\begin{figure}[h]
\centerline{\fbox{\psfig{figure=figures/deformation2d.ps,width=0.70\textwidth}}}
\caption{2D element before and after deformation~\cite{Logan07}.}
\label{fig:deformation2d}
\end{figure}

\begin{equation}
\varepsilon_{x} = \frac{A'B' - AB}{AB}
\end{equation}

\noindent and

\begin{equation}
AB = dx.
\end{equation}

\noindent The elemental edge $dx$ that is initially parallel to the x-axis is deformed as $(A'B')^2$ (cf. Equation~\ref{eqn:3.5}). The final length can be evaluated by


\begin{equation}
\begin{array}{l}
(A'B')^2  = {\left(dx +  \frac{\partial{u_{x}}}{\partial{x}}dx \right)}^2 + {\left(\frac{\partial{u_{y}}}{\partial{x}}dx\right)}^2, \\
(A'B')^2  = dx
\begin{bmatrix}
1 + 2{\left(\frac{\partial{u_{x}}}{\partial{x}} \right)} + {\left(\frac{\partial{u_{x}}}{\partial{x}} \right)}^2 + {\left(\frac{\partial{u_{y}}}{\partial{x}}\right)}^2
 \end{bmatrix}.
\end{array}
\label{eqn:3.5}
\end{equation}

\noindent By neglecting the higher order terms in Equation~\ref{eqn:3.5}, the 2D strains are defined by

\begin{equation}
\begin{array}{l}
\varepsilon_{xx} = \frac{\partial{u_{x}}}{\partial{x}} \\
\varepsilon_{yy} = \frac{\partial{u_{y}}}{\partial{y}} \\
\gamma_{xy} = \frac{1}{2}\left(\frac{\partial{u_{x}}}{\partial{y}} + \frac{\partial{u_{y}}}{\partial{x}}\right)
\end{array}
\end{equation}

\noindent After finding the 2D strains, it is straightforward to expand it to the 3D case by

\begin{equation}
\begin{array}{l}
\varepsilon_{zz} = \frac{\partial{u_{z}}}{\partial{z}} \\
\gamma_{zx} = \frac{1}{2}\left(\frac{\partial{u_{z}}}{\partial{x}} + \frac{\partial{u_{x}}}{\partial{z}}\right) \\
\gamma_{yz} = \frac{1}{2}\left(\frac{\partial{u_{y}}}{\partial{z}} + \frac{\partial{u_{z}}}{\partial{y}}\right)
\end{array}
\end{equation}

\noindent Infinitesimal strains with 3D element are given by

\begin{equation}
\{\varepsilon\}
=
\begin{Bmatrix}
\varepsilon_{xx} \\
\varepsilon_{yy} \\
\varepsilon_{zz} \\
2\gamma_{xy} \\
2\gamma_{zx} \\
2\gamma_{yz}
\end{Bmatrix}
=
\begin{Bmatrix}
\frac{\partial{u_{x}}}{\partial{x}} \\
\frac{\partial{u_{y}}}{\partial{y}} \\
\frac{\partial{u_{z}}}{\partial{z}} \\
\frac{\partial{u_{x}}}{\partial{y}} + \frac{\partial{u_{y}}}{\partial{x}} \\
\frac{\partial{u_{z}}}{\partial{x}} + \frac{\partial{u_{x}}}{\partial{z}} \\
\frac{\partial{u_{y}}}{\partial{z}} + \frac{\partial{u_{z}}}{\partial{y}}
\end{Bmatrix}.
\end{equation}

\noindent After finding strains, these equations are combined with shape functions to find matrix $[B]$:


\begin{equation}
\begin{array}{l}
\{\varepsilon\} = [B]\{d\}.
\end{array}
\label{eqn:3.6}
\end{equation}

\noindent Using Equations~\ref{eqn:3.4} for displacements, we can evaluate the partial derivatives of the shape functions as follows:

\begin{equation}
\begin{array}{l}
\frac{\partial{u_{x}}}{\partial{x}}
=
\frac{\partial{}}{\partial{x}}(\psi_{1} + \psi_{2} + \psi_{3} + \psi_{4})
=
\frac{1}{6V}(\beta_{1}u_{1} + \beta_{2}u_{2} + \beta_{3}u_{3} + \beta_{4}u_{4}) \\
\frac{\partial{u_{y}}}{\partial{y}}
=
\frac{1}{6V}(\gamma_{1}v_{1} + \gamma_{2}v_{2} + \gamma_{3}v_{3} + \gamma_{4}v_{4}) \\
\frac{\partial{u_{z}}}{\partial{z}}
=
\frac{1}{6V}(\delta_{1}w_{1} + \delta_{2}w_{2} + \delta_{3}w_{3} + \delta_{4}w_{4}) \\
\frac{\partial{u_{x}}}{\partial{y}} + \frac{\partial{u_{y}}}{\partial{x}}
=
\frac{1}{6V}(\gamma_{1}u_{1} + \gamma_{2}u_{2} + \gamma_{3}u_{3} + \gamma_{4}u_{4} + \beta_{1}v_{1} + \beta_{2}v_{2} + \beta_{3}v_{3} + \beta_{4}v_{4}) \\
\frac{\partial{u_{z}}}{\partial{x}} + \frac{\partial{u_{x}}}{\partial{z}}
=
\frac{1}{6V}(\beta_{1}w_{1} + \beta_{2}w_{2} + \beta_{3}w_{3} + \beta_{4}w_{4} + \delta_{1}u_{1} + \delta_{2}u_{2} + \delta_{3}u_{3} + \delta_{4}u_{4}) \\
\frac{\partial{u_{y}}}{\partial{z}} + \frac{\partial{u_{z}}}{\partial{y}}
=
\frac{1}{6V}(\delta_{1}v_{1} + \delta_{2}v_{2} + \delta_{3}v_{3} + \delta_{4}v_{4} + \gamma_{1}w_{1} + \gamma_{2}w_{2} + \gamma_{3}w_{3} + \gamma_{4}w_{4})
\end{array}
\label{eqn:3.601}
\end{equation}

\noindent Using Equations~\ref{eqn:3.601} for the 1$^{st}$ node, we obtain the submatrix $[B_1]$ of $[B]$ as
\begin{equation}
[B_{1}]
 =
\begin{bmatrix}
    {\frac{\partial{u_{x}}}{\partial{x}}} & 0 & 0 \\
    0 & \frac{\partial{u_{y}}}{\partial{y}} & 0 \\
    0  & 0  & \frac{\partial{u_{z}}}{\partial{z}}  \\
    \frac{\partial{u_{x}}}{\partial{y}} & \frac{\partial{u_{y}}}{\partial{x}} & 0 \\
    \frac{\partial{u_{z}}}{\partial{x}} & 0 & \frac{\partial{u_{x}}}{\partial{z}} \\
    0 &  \frac{\partial{u_{y}}}{\partial{z}} & \frac{\partial{u_{z}}}{\partial{y}}
 \end{bmatrix}
\begin{Bmatrix}
u_{1} \\
v_{1} \\
w_{1}
\end{Bmatrix}.
 \label{eqn:3.7}
\end{equation}

\noindent Using Equations~\ref{eqn:3.601}~and~\ref{eqn:3.7}, $\{\varepsilon\} $ can be written as

\begin{equation}
\{\varepsilon\} = [B]\{d\}
 =
\setcounter{MaxMatrixCols}{20}
\begin{bmatrix}
    \beta_{1} & 0 & 0 & \beta_{2} & 0 & 0 &\beta_{3} & 0 & 0 & \beta_{4} & 0 & 0 \\
    0 & \gamma_{1} & 0 & 0 & \gamma_{2} & 0 & 0 & \gamma_{3} & 0 & 0 & \gamma_{4} & 0 \\
    0  & 0  & \delta_{1} & 0  & 0  & \delta_{2} & 0  & 0  & \delta_{3} & 0  & 0  & \delta_{4} \\
    \gamma_{1} & \beta_{1} & 0 & \gamma_{2} & \beta_{2} & 0 & \gamma_{3} & \beta_{3} & 0 & \gamma_{4} & \beta_{4} & 0 \\
    \delta_{1} & 0 & \beta_{1} & \delta_{2} & 0 & \beta_{2} & \delta_{3} & 0 & \beta_{3} & \delta_{4} & 0 & \beta_{4}   \\
    0 &  \delta_{1} & \gamma_{1} & 0 &  \delta_{2} & \gamma_{2} & 0 &  \delta_{3} & \gamma_{3} & 0 &  \delta_{5} & \gamma_{5}
 \end{bmatrix}
 \begin{Bmatrix} u_{1} \\ v_{1} \\ w_{1} \\ u_{2} \\ v_{2} \\ w_{2} \\ u_{3} \\ v_{3} \\ w_{3}\\ u_{4} \\ v_{4} \\ w_{4} \end{Bmatrix}.
 \label{eqn:3.71}
\end{equation}

\noindent In Equation~\ref{eqn:1.1}, the engineering stress vector $\{ \sigma \}$ is related to the strain vector $\{ \varepsilon \}$ by

\begin{equation}
\begin{array}{l}
\{\sigma\} = [E]\{\varepsilon\}, \\
\{\sigma\} = [E][B]\{d\},
\label{eqn:3.8}
\end{array}
\end{equation}

\noindent where $[E]$ is the material property matrix (constitutive matrix) defined by

\begin{equation}
[E]
 =
 \frac{\epsilon}{(1 + \nu)(1 - 2\nu)}
\begin{bmatrix}
(1 - \nu) & \nu & \nu & 0 & 0 & 0 \\
\nu & (1 - \nu) & \nu & 0 & 0 & 0 \\
\nu & \nu & (1 - \nu) & 0 & 0 & 0 \\
0 & 0 & 0 & \frac{(1 - 2\nu)}{2} & 0 & 0 \\
0 & 0 & 0 & 0 & \frac{(1 - 2\nu)}{2} & 0 \\
0 & 0 & 0 & 0 & 0 & \frac{(1 - 2\nu)}{2}
\end{bmatrix},
\label{eqn:EMatrix}
\end{equation}

\noindent where $\epsilon$ is the Young's modulus and $\nu$ is the Poisson's ratio. Young's modulus describes the elastic properties of a solid undergoing tension or compression. Poisson's ratio is the ratio of transverse strain (perpendicular to the applied load), to the longitudinal strain (in the direction of the applied load)~\cite{Modulus2011}. From the conservation of the potential energy, substituting Equations~\ref{eqn:3.6}~and~\ref{eqn:3.8} into Equation~\ref{eqn:1.1}, we obtain the element stiffness matrix


\begin{equation}
[k] = \int\int\int\{d\}^T[B]^T[E][B]\{d\} dx \; dy \; dz.
\label{eqn:3.9}
\end{equation}

\noindent As seen from Equations~\ref{eqn:3.7}~and~\ref{eqn:EMatrix}, the matrices $[B]$ and $[E]$ are constant for a tetrahedral element, so that Equation~\ref{eqn:3.9} is rewritten as

\begin{equation}
[k] = \{d\}^T[B]^T[E][B]\{d\} V .
\label{eqn:3.10}
\end{equation}

\noindent With the introduction of the nodal forces per element,

\begin{equation}
\begin{Bmatrix}
 f
 \end{Bmatrix}
 =
\begin{Bmatrix}
f_{1x} \\ f_{1y} \\ f_{1z} \\ \vdots \\ f_{4x} \\ f_{4y} \\ f_{4z}
\end{Bmatrix}
\{d\}^T.
\label{eqn:3.11}
\end{equation}


\noindent With the equilibrium equation and the cancellation of the $\{d\}^T$, the whole system for one element reduces to

\begin{equation}
K^e \{d\}^e = \{f\}^e.
\label{eqn:3.12}
\end{equation}

\noindent By substituting $\{d\}$ with $u$, we obtain~\cite{Logan07}:

\begin{equation}
K^e u^e = f^e.
\label{eqn:3.13}
\end{equation}

\subsection{Assembly of Elemental Stiffness Matrices}

\begin{figure}[h]
\centerline{\fbox{\psfig{figure=figures/assemble.ps,width=0.35\textwidth}}}
\caption{A tetrahedral mesh with two elements}
\label{fig:asmbl}
\end{figure}

We apply assembly process using the element's nodal information (which nodes belong to which elements). The size of elemental stiffness matrix is $12 \times 12$ (the tetrahedron has four nodes and it is three-dimensional). The size of global stiffness matrix is $3N \times 3N$, where $N$ is the total number of nodes of the whole system.
In order to complete the assembly process, all elemental stiffness matrices must be copied to the correct index of the global stiffness matrix. The assembly operation is described by using two elements (Figure~\ref{fig:asmbl}) as an example. The stiffness matrices of the first and second elements are given as

\begin{equation}
K_1
 =
\begin{bmatrix}
k^{1}_{1,1} & k^{1}_{1,2} & k^{1}_{1,3} & \hdots & k^{1}_{1,12} \\
k^{1}_{2,1} & k^{1}_{2,2} & k^{1}_{2,3} & \hdots & k^{1}_{2,12} \\
k^{1}_{3,1} & k^{1}_{3,2} & k^{1}_{3,3} & \hdots & k^{1}_{3,12} \\
\vdots  & \vdots  & \vdots  & \vdots & \vdots   \\
k^{1}_{12,1} & k^{1}_{12,2} & k^{1}_{12,3} & \hdots & k^{1}_{12,12}
\end{bmatrix} \;\;\text{and}\;\;
K_2
 =
\begin{bmatrix}
k^{2}_{1,1} & k^{2}_{1,2} & k^{2}_{1,3} & \hdots & k^{2}_{1,12} \\
k^{2}_{2,1} & k^{2}_{2,2} & k^{2}_{2,3} & \hdots & k^{2}_{2,12} \\
k^{2}_{3,1} & k^{2}_{3,2} & k^{2}_{3,3} & \hdots & k^{2}_{3,12} \\
\vdots  & \vdots  & \vdots  & \vdots & \vdots   \\
k^{2}_{12,1} & k^{2}_{12,2} & k^{2}_{12,3} & \hdots & k^{2}_{12,12}
\end{bmatrix}.
\label{eqn:element1_2}
\end{equation}

It can be seen from Figure~\ref{fig:asmbl} that Nodes 2, 3 and 4 belong to both Element~1 and Element~2. When we assemble the elements, these shared values in the global stiffness matrix (5 nodes, $15 \times 15$ matrix) come from both Element~1 and Element~2. The elements are assembled using Algorithm~\ref{alg:assemblealgorithm}, which constructs the global stiffness matrix~$K$ (see~Equation~\ref{eqn:gbmatrix}).

\begin{algorithm}                      % enter the algorithm environment
\caption{Assembly of the Elements}     % give the algorithm a caption
\label{alg:assemblealgorithm}          % and a label for \ref{} commands later in the document
{
\fontsize{10}{10}\selectfont
\begin{algorithmic}[l]
\FOR{i = 1 to N}  \vspace*{1.0ex}
     \STATE $f_{BI}$ = (($1^{st}$ node of $i^{th}$ element - 1) $\times$ 3) + 1 \vspace*{1.0ex}
     \STATE $s_{BI}$ = (($2^{nd}$ node of $i^{th}$ element - 1) $\times$ 3) + 1 \vspace*{1.0ex}
     \STATE $t_{BI}$ = (($3^{rd}$ node of $i^{th}$ element - 1) $\times$ 3) + 1 \vspace*{1.0ex}
     \STATE $r_{BI}$ = (($4^{th}$ node of $i^{th}$ element - 1) $\times$ 3) + 1 \vspace*{1.0ex}
     \STATE $K$[$f_{BI}$:$f_{BI}$+2, $f_{BI}$: $f_{BI}$+2] += $K_{i}$[1:3,1:3] \vspace*{1.0ex}
     \STATE $K$[$f_{BI}$:$f_{BI}$+2, $s_{BI}$: $s_{BI}$+2] += $K_{i}$[1:3,4:6] \vspace*{1.0ex}
     \STATE $K$[$f_{BI}$:$f_{BI}$+2, $t_{BI}$: $t_{BI}$+2] += $K_{i}$[1:3,7:9] \vspace*{1.0ex}
     \STATE $K$[$f_{BI}$:$f_{BI}$+2, $r_{BI}$: $r_{BI}$+2] = $K_{i}$[1:3,10:12] \vspace*{1.0ex}
     \STATE $K$[$s_{BI}$:$s_{BI}$+2, $f_{BI}$: $f_{BI}$+2] += $K_{i}$[4:6,1:3] \vspace*{1.0ex}
     \STATE $K$[$s_{BI}$:$s_{BI}$+2, $s_{BI}$: $s_{BI}$+2] += $K_{i}$[4:6,4:6] \vspace*{1.0ex}
     \STATE $K$[$s_{BI}$:$s_{BI}$+2, $t_{BI}$: $t_{BI}$+2] += $K_{i}$[4:6,7:9] \vspace*{1.0ex}
     \STATE $K$[$s_{BI}$:$s_{BI}$+2, $r_{BI}$: $r_{BI}$+2] += $K_{i}$[4:6,10:12] \vspace*{1.0ex}
     \STATE $K$[$t_{BI}$:$t_{BI}$+2, $f_{BI}$: $f_{BI}$+2] += $K_{i}$[7:9,1:3] \vspace*{1.0ex}
     \STATE $K$[$t_{BI}$:$t_{BI}$+2, $s_{BI}$: $s_{BI}$+2] += $K_{i}$[7:9,4:6] \vspace*{1.0ex}
     \STATE $K$[$t_{BI}$:$t_{BI}$+2, $t_{BI}$: $t_{BI}$+2] += $K_{i}$[7:9,7:9] \vspace*{1.0ex}
     \STATE $K$[$t_{BI}$:$t_{BI}$+2, $r_{BI}$: $r_{BI}$+2] += $K_{i}$[7:9,10:12] \vspace*{1.0ex}
     \STATE $K$[$r_{BI}$:$r_{BI}$+2, $r_{BI}$: $r_{BI}$+2] += $K_{i}$[10:12,1:3] \vspace*{1.0ex}
     \STATE $K$[$r_{BI}$:$r_{BI}$+2, $s_{BI}$: $s_{BI}$+2] += $K_{i}$[10:12,4:6] \vspace*{1.0ex}
     \STATE $K$[$r_{BI}$:$r_{BI}$+2, $t_{BI}$: $t_{BI}$+2] += $K_{i}$[10:12,7:9] \vspace*{1.0ex}
    \STATE $K$[$r_{BI}$:$r_{BI}$+2, $r_{BI}$: $r_{BI}$+2] += $K_{i}$[10:12,10:12] \vspace*{1.0ex}	
\ENDFOR

\end{algorithmic}
}
\end{algorithm}

\begin{landscape}
\begin{equation}
\begin{array}{l}
K =
\left[ \begin{smallmatrix} \\ \vspace*{1.0ex}
k^{1}_{1,1} & k^{1}_{1,2} & k^{1}_{1,3} & k^{1}_{1,4} & k^{1}_{1,5} & k^{1}_{1,6} & k^{1}_{1,7} & k^{1}_{1,8} & k^{1}_{1,9} & k^{1}_{1,10} & k^{1}_{1,11} & k^{1}_{1,12} & 0 & 0 & 0 \\ \vspace*{1.0ex}
k^{1}_{2,1} & k^{1}_{2,2} & k^{1}_{2,3} & k^{1}_{2,4} & k^{1}_{2,5} & k^{1}_{2,6} & k^{1}_{2,7} & k^{1}_{2,8} & k^{1}_{2,9} & k^{1}_{2,10} & k^{1}_{2,11} & k^{1}_{2,12} & 0 & 0 & 0 \\ \vspace*{1.0ex}
k^{1}_{3,1} & k^{1}_{3,2} & k^{1}_{3,3} & k^{1}_{3,4} & k^{1}_{3,5} & k^{1}_{3,6} & k^{1}_{3,7} & k^{1}_{3,8} & k^{1}_{3,9} & k^{1}_{3,10} & k^{1}_{3,11} & k^{1}_{3,12} & 0 & 0 & 0 \\ \vspace*{1.0ex}
k^{1}_{4,1} & k^{1}_{4,2} & k^{1}_{4,3} & k^{1}_{4,4} + k^{2}_{1,1} & k^{1}_{4,5} + k^{2}_{1,2} & k^{1}_{4,6} + k^{2}_{1,3} & k^{1}_{4,7} + k^{2}_{1,4} & k^{1}_{4,8} + k^{2}_{1,5} & k^{1}_{4,9} + k^{2}_{1,6} & k^{1}_{4,10} + k^{2}_{1,7} & k^{1}_{4,11} + k^{2}_{1,8} & k^{1}_{4,12} + k^{2}_{1,9} & k^{2}_{1,10} & k^{2}_{1,11} & k^{2}_{1,12} \\ \vspace*{1.0ex}
k^{1}_{5,1} & k^{1}_{5,2} & k^{1}_{5,3} & k^{1}_{5,4} + k^{2}_{2,1} & k^{1}_{5,5} + k^{2}_{2,2} & k^{1}_{5,6} + k^{2}_{2,3} & k^{1}_{5,7} + k^{2}_{2,4} & k^{1}_{5,8} + k^{2}_{2,5} & k^{1}_{5,9} + k^{2}_{2,6} & k^{1}_{5,10} + k^{2}_{2,7} & k^{1}_{5,11} + k^{2}_{2,8} & k^{1}_{5,12} + k^{2}_{2,9} & k^{2}_{2,10} & k^{2}_{2,11} & k^{2}_{2,12} \\ \vspace*{1.0ex}
k^{1}_{6,1} & k^{1}_{6,2} & k^{1}_{6,3} & k^{1}_{6,4} + k^{2}_{3,1} & k^{1}_{6,5} + k^{2}_{3,2} & k^{1}_{6,6} + k^{2}_{3,3} & k^{1}_{6,7} + k^{2}_{3,4} & k^{1}_{6,8} + k^{2}_{3,5} & k^{1}_{6,9} + k^{2}_{3,6} & k^{1}_{6,10} + k^{2}_{3,7} & k^{1}_{6,11} + k^{2}_{3,8} & k^{1}_{6,12} + k^{2}_{3,9} & k^{2}_{3,10} & k^{2}_{3,11} & k^{2}_{3,12} \\ \vspace*{1.0ex}
k^{1}_{7,1} & k^{1}_{7,2} & k^{1}_{7,3} & k^{1}_{7,4} + k^{2}_{4,1} & k^{1}_{7,5} + k^{2}_{4,2} & k^{1}_{7,6} + k^{2}_{4,3} & k^{1}_{7,7} + k^{2}_{4,4} & k^{1}_{7,8} + k^{2}_{4,5} & k^{1}_{7,9} + k^{2}_{4,6} & k^{1}_{7,10} + k^{2}_{4,7} & k^{1}_{7,11} + k^{2}_{4,8} & k^{1}_{7,12} + k^{2}_{4,9} & k^{2}_{4,10} & k^{2}_{4,11} & k^{2}_{4,12} \\ \vspace*{1.0ex}
k^{1}_{8,1} & k^{1}_{8,2} & k^{1}_{8,3} & k^{1}_{8,4} + k^{2}_{5,1} & k^{1}_{8,5} + k^{2}_{5,2} & k^{1}_{8,6} + k^{2}_{5,3} & k^{1}_{8,7} + k^{2}_{5,4} & k^{1}_{8,8} + k^{2}_{5,5} & k^{1}_{8,9} + k^{2}_{5,6} & k^{1}_{8,10} + k^{2}_{5,7} & k^{1}_{8,11} + k^{2}_{5,8} & k^{1}_{8,12} + k^{2}_{5,9} & k^{2}_{5,10} & k^{2}_{5,11} & k^{2}_{5,12} \\ \vspace*{1.0ex}
k^{1}_{9,1} & k^{1}_{9,2} & k^{1}_{9,3} & k^{1}_{9,4} + k^{2}_{6,1} & k^{1}_{9,5} + k^{2}_{6,2} & k^{1}_{9,6} + k^{2}_{6,3} & k^{1}_{9,7} + k^{2}_{6,4} & k^{1}_{9,8} + k^{2}_{6,5} & k^{1}_{9,9} + k^{2}_{6,6} & k^{1}_{9,10} + k^{2}_{6,7} & k^{1}_{9,11} + k^{2}_{6,8} & k^{1}_{9,12} + k^{2}_{6,9} & k^{2}_{6,10} & k^{2}_{6,11} & k^{2}_{6,12} \\ \vspace*{1.0ex}
k^{1}_{10,1} & k^{1}_{10,2} & k^{1}_{10,3} & k^{1}_{10,4} + k^{2}_{7,1} & k^{1}_{10,5} + k^{2}_{7,2} & k^{1}_{10,6} + k^{2}_{7,3} & k^{1}_{10,7} + k^{2}_{7,4} & k^{1}_{10,8} + k^{2}_{7,5} & k^{1}_{10,9} + k^{2}_{7,6} & k^{1}_{10,10} + k^{2}_{7,7} & k^{1}_{10,11} + k^{2}_{7,8} & k^{1}_{10,12} + k^{2}_{7,9} & k^{2}_{7,10} & k^{2}_{7,11} & k^{2}_{7,12} \\ \vspace*{1.0ex}
k^{1}_{11,1} & k^{1}_{11,2} & k^{1}_{11,3} & k^{1}_{11,4} + k^{2}_{8,1} & k^{1}_{11,5} + k^{2}_{8,2} & k^{1}_{11,6} + k^{2}_{8,3} & k^{1}_{11,7} + k^{2}_{8,4} & k^{1}_{11,8} + k^{2}_{8,5} & k^{1}_{11,9} + k^{2}_{8,6} & k^{1}_{11,10} + k^{2}_{8,7} & k^{1}_{11,11} + k^{2}_{8,8} & k^{1}_{11,12} + k^{2}_{8,9} & k^{2}_{8,10} & k^{2}_{8,11} & k^{2}_{8,12} \\ \vspace*{1.0ex}
k^{1}_{12,1} & k^{1}_{12,2} & k^{1}_{12,3} & k^{1}_{12,4} + k^{2}_{9,1} & k^{1}_{12,5} + k^{2}_{9,2} & k^{1}_{12,6} + k^{2}_{9,3} & k^{1}_{12,7} + k^{2}_{9,4} & k^{1}_{12,8} + k^{2}_{9,5} & k^{1}_{12,9} + k^{2}_{9,6} & k^{1}_{12,10} + k^{2}_{9,7} & k^{1}_{12,11} + k^{2}_{9,8} & k^{1}_{12,12} + k^{2}_{9,9} & k^{2}_{9,10} & k^{2}_{9,11} & k^{2}_{9,12} \\ \vspace*{1.0ex}
0 & 0 & 0 & k^{2}_{10,1} & k^{2}_{10,2} & k^{2}_{10,3} & k^{2}_{10,4} & k^{2}_{10,5} & k^{2}_{10,6} & k^{2}_{10,7} & k^{2}_{10,8} & k^{2}_{10,9} & k^{2}_{10,10} & k^{2}_{10,11} & k^{2}_{10,12} \\ \vspace*{1.0ex}
0 & 0 & 0 & k^{2}_{11,1} & k^{2}_{11,2} & k^{2}_{11,3} & k^{2}_{11,4} & k^{2}_{11,5} & k^{2}_{11,6} & k^{2}_{11,7} & k^{2}_{11,8} & k^{2}_{11,9} & k^{2}_{11,10} & k^{2}_{11,11} & k^{2}_{11,12} \\ \vspace*{1.0ex}
0 & 0 & 0 & k^{2}_{12,1} & k^{2}_{12,2} & k^{2}_{12,3} & k^{2}_{12,4} & k^{2}_{12,5} & k^{2}_{12,6} & k^{2}_{12,7} & k^{2}_{12,8} & k^{2}_{12,9} & k^{2}_{12,10} & k^{2}_{12,11} & k^{2}_{12,12} \\
\end{smallmatrix} \right] \\
\end{array}.
\label{eqn:gbmatrix}
\end{equation}
\end{landscape}

\subsection{Applying Boundary Conditions}
After assembling the elemental stiffness matrices and nodal force vectors, boundary conditions are applied by assigning $1s$ and $0s$ to the corresponding rows and columns according to constrained nodes by Algorithm~\ref{eqn:boundaryalgorithm}.

\begin{algorithm}                      % enter the algorithm environment
\caption{Boundary Value Assignment}    % give the algorithm a caption
\label{eqn:boundaryalgorithm}          % and a label for \ref{} commands later in the document
{
\fontsize{10}{10}\selectfont
\begin{algorithmic}[l]
\FOR{i  = 1 to  BC (BC is the number of constrained nodes)} \vspace*{1.0ex}
     \STATE BI is (($i^{th}$ constrained node - 1) $\times$ 3) + 1 \vspace*{1.0ex}
     \STATE K[BI: BI + 2, 1: dimension] = 0                \vspace*{1.0ex}
     \STATE K[1: dimension, BI: BI + 2] = 0                \vspace*{1.0ex}
     \STATE K[BI, BI] = 1                                    \vspace*{1.0ex}
     \STATE K[BI+1, BI+1] = 1                                \vspace*{1.0ex}
     \STATE K[BI+2, BI+2] = 1                                \vspace*{1.0ex}
     \STATE F[BI: BI + 2] = 0                               \vspace*{1.0ex}

\ENDFOR

\end{algorithmic}
}
\end{algorithm}

\noindent The system in Figure~\ref{fig:asmbl} is constrained from nodes 1, 2, and 3. Algorithm~\ref{eqn:boundaryalgorithm} is used to obtain the global stiffness matrix $K$:

\begin{equation}
K=
\left[ \begin{smallmatrix} \  & & & & & & & & & & & & & & \\ \vspace*{1.0ex}
1 &	\ 0 \ &	\ 0 \ &	\ 0 \ &	\ 0 \ &	\ 0 \ &	\ 0 \ &	\ 0 \ &	\ 0 \ &	\ 0 \ &	\ 0 \ &	\ 0 \ &	\ 0 \ &	\ 0 \ &	0 \\	\vspace*{1.0ex}
0 &	1 &	0 &	0 &	0 &	0 &	0 &	0 &	0 &	0 &	0 &	0 &	0 &	0 &	0 \\	\vspace*{1.0ex}
0 &	0 &	1 &	0 &	0 &	0 &	0 &	0 &	0 &	0 &	0 &	0 &	0 &	0 &	0 \\	\vspace*{1.0ex}
0 &	0 &	0 &	1 &	0 &	0 &	0 &	0 &	0 &	0 &	0 &	0 &	0 &	0 &	0 \\	\vspace*{1.0ex}
0 &	0 &	0 &	0 &	1 &	0 &	0 &	0 &	0 &	0 &	0 &	0 &	0 &	0 &	0 \\	\vspace*{1.0ex}
0 &	0 &	0 &	0 &	0 &	1 &	0 &	0 &	0 &	0 &	0 &	0 &	0 &	0 &	0 \\	\vspace*{1.0ex}
0 &	0 &	0 &	0 &	0 &	0 &	1 &	0 &	0 &	0 &	0 &	0 &	0 &	0 &	0 \\    \vspace*{1.0ex}
0 &	0 &	0 &	0 &	0 &	0 &	0 &	1 &	0 &	0 &	0 &	0 &	0 &	0 &	0 \\    \vspace*{1.0ex}
0 &	0 &	0 &	0 &	0 &	0 &	0 &	0 &	1 &	0 &	0 &	0 &	0 &	0 &	0 \\	\vspace*{1.0ex}
0 &	0 &	0 &	0 &	0 &	0 &	0 &	0 &	0 &	k^{1}_{10,10} + k^{2}_{7,7}	k^{1}_{10,10} + k^{2}_{7,7} & k^{1}_{10,11} + k^{2}_{7,8} & k^{1}_{10,12} + k^{2}_{7,9} & k^{2}_{7,10} & k^{2}_{7,11} & k^{2}_{7,12} \\ \vspace*{1.0ex}
0 &	0 &	0 &	0 &	0 &	0 &	0 &	0 &	0 &	k^{1}_{11,10} + k^{2}_{8,7} & k^{1}_{11,11} + k^{2}_{8,8} & k^{1}_{11,12} + k^{2}_{8,9} & k^{2}_{8,10} & k^{2}_{8,11} & k^{2}_{8,12} \\                             \vspace*{1.0ex}
0 &	0 &	0 &	0 &	0 &	0 &	0 &	0 &	0 &	k^{1}_{12,10} + k^{2}_{9,7} & k^{1}_{12,11} + k^{2}_{9,8} & k^{1}_{12,12} + k^{2}_{9,9} & k^{2}_{9,10} & k^{2}_{9,11} & k^{2}_{9,12} \\                             \vspace*{1.0ex}
0 &	0 &	0 &	0 &	0 &	0 &	0 &	0 &	0 &	k^{2}_{10,7} & k^{2}_{10,8} & k^{2}_{10,9} & k^{2}_{10,10} & k^{2}_{10,11} & k^{2}_{10,12} \\                                                                       \vspace*{1.0ex}
0 &	0 &	0 &	0 &	0 &	0 &	0 &	0 &	0 &	k^{2}_{11,7} & k^{2}_{11,8} & k^{2}_{11,9} & k^{2}_{11,10} & k^{2}_{11,11} & k^{2}_{11,12} \\                                                                       \vspace*{1.0ex}
0 &	0 &	0 &	0 &	0 &	0 &	0 &	0 &	0 &	k^{2}_{12,7} & k^{2}_{12,8} & k^{2}_{12,9} & k^{2}_{12,10} & k^{2}_{12,11} & k^{2}_{12,12}
\end{smallmatrix} \right].
\label{eqn:gbmatrixboundary}
\end{equation}

\subsection{Solution of the Linear System}

\noindent After applying boundary conditions to the elemental stiffness matrices and nodal force vectors, the whole system is one large linear system:

\begin{equation}
K u = f.
\label{eqn:3.14}
\end{equation}

\noindent In the last step, solving the system gives unknown nodal displacements

\begin{equation}
u = K^{-1}f.
\label{eqn:3.15}
\end{equation}


