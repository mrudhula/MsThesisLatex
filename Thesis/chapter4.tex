\chapter{Collision}
\label{chapter_collision}

Currently, the collision between the cloth and the body parts and the self-collision of cloth particles are implemented. This collision is handled by the PhysX engine. 

\section{Preparation and Settings}
\label{section_collision_preparation}

Cloth vertices can collide with pre-defined spheres, capsules or planes. In order to keep the program stable and running at real-time, it is important to find the balance between collision detail and performance. 

The female body bone information is extracted from the input skeletal mesh class, and the colliding capsules are generated automatically, although the radii of the bones are specified manually. The collision information is specified in two arrays, one being the positions and radii of the spheres and the second defining pairs of spheres which form capsules. A total of 16 capsules and spheres are used in the whole collision model, which simulates all the movable bones of the female body (Figure \ref{fig:colliding_human_body}). The collision data is prepared before the cloth creation, and given as a parameter. 

\begin{figure}[h]
\centerline{\fbox{\psfig{figure=figures/colliding_human_body.png,width=1.00\textwidth}}}
\caption{Character formed with collision spheres and capsules}
\label{fig:colliding_human_body}
\end{figure}

Other than the defined collision spheres and capsules, the cloth naturally collides with the floor actor of the PhysX environment as well. This however, introduces a problem: The models from Blender are exported as root joint coinciding with the origin. This is required for successful animation. However, in the PhysX environment, the floor is created automatically at (0,0,0), and the initial position of the lower dress is partly below floor, which produces unrealistic visuals. In order to overcome this issue, the lower cloth vertices are introduced into the software with increased y coordinates, in order to keep them above ground. In the rendering process, lower cloth is attached to a scene node with negative y offset equal to the boost in the vertices, which places the lower dress exactly where it should be. This process overcomes the floor collision problem.

Every frame, the collision sphere positions are updated with the data from the updated body skeleton, prior to the cloth simulation. 

\section{Algorithm}

PhysX provides two options for collision detection: Discrete and Continuous Collision. I chose to work with the latter due to more robust results, although it takes twice as long to do the calculation.

The solution is performed for the trajectory of the capsule or sphere and the particle for frame interval. This approach is especially robust in fast motion, which is important since the motion is created in real time. The required solution is a 6th degree polynomial, which is approximated with quadratic equation. The pipeline is given in Figure \ref{fig:collision_pipeline}.

The solution is performed on the GPU, which increases the parallel performance greatly, allowing for frame rates of 600+fps. The cloth is discretized as a triangle mesh, and since the collision is only detected on vertices, the density must be high enough in order to avoid penetration at the areas \cite{Kim2011}, \cite{Tonge2010}. 

\begin{figure}[h]
\centerline{\fbox{\psfig{figure=figures/collision_pipeline.png,width=1.00\textwidth}}}
\caption{Collision Detection Pipeline \cite{Tonge2010} }
\label{fig:collision_pipeline}
\end{figure}