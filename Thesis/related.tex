\chapter{Related Work}
\label{chapter_related_work}

Virtual fitting frameworks are complex systems composed of many modules. The related works on the major components of our framework are summarized in this chapter, 
along with the history of the general framework. 

The overall framework can be divided into three major modules: \textit{human body animation and motion smoothing} , \textit{motion capture systems and depth sensors} and 
\textit{physics engines, cloth simulation and collision detection}. An important offline part of the framework is the \textit{modeling humans and apparels}, which has evolved 
significantly over the last years. This chapter will start by exploring the modeling aspect of the framework, continued by the anaylsis of runtime components.

\section{Human and Apparel Modeling}
\label{section_related_modeling}

Although the human and apparel modeling both have foundations in 3D mesh creation, they require different traits and qualities. Human body modeling utilizes disciplines such as rigging, 
skinning and multi-layered deformation, whereas apparel modeling is mostly based on physics simulations.

\subsection{Human Body Modeling}
As we humans are mostly the main characters in virtual worlds, very different methods and techniques exist for human body models and animations. Determining the suitable one 
starts with the requirements of the application. Real-time applications require a certain level of simplicity, as the simulation time cannot exceed the frame duration. Offline 
applications can utilize very higher detail models, looking much more realistic. The basis for both extremes however, are the same, which is a skeleton. The approach is starting 
with a connected set of rigid objects named as \textit{bones}, continuing by adding layers of muscle, skin, hair and others, depending on the required quality level. Called \textit{layered
modeling technique}, this is a very common modeling technique used in computer graphics \cite{Chadwick1989}. The animation is achieved by rotating the bones, which are followed by 
upper layers. This technique also improves the reusability of the framework, as the same animation sequence can be used for multiple body models with different detail levels utilizing the 
same base skeleton.   

The articulated skeleton consists of a hierarchical structure of joints and limbs to model a human-like skeleton. Joints are the points which act as the origin of the respective 
local coordinate space. The limbs are the rigid segments which connect the joints in the hierarchy. Rotation in the local coordinate systems defined by the joints cause the rigid
limbs to be rotated, resulting in the motion. The complexity of the model can be determined by the number of joints and the degrees of freedoms - abbreviated as \textit{DOF}s.
Degree of freedom is defined as ``the number of independent parameters that define its configuration''\cite{Lazard2013}. A joint can rotate and translate in at most three orthagonal
directions, hence the maximum DOF a joint can have is six. Although having the maximum number of DOFs in a human body model might seem like a good way to improve the realism, 
however this also increases the complexity of the structure, resulting in more mathematical operations. As most human-body joints can only rotate, not translate, assigning six 
degrees of freedom to every joint is redundant. Furthermore, angular and axis constraints with certain joints (such as knee or elbow) further simplify the model while making it 
more realistic. 
 



\section{Human Body Animation}
\label{section_related_human_body_animation}

