\chapter{3-D Model}
\label{chapter_3d_model}

Two sets of 3D meshes are required in order to perform simulation: 
\begin{itemize}
\item A Cloth Mesh to be processed by the embedded physics engine.
\item A Full Body Human Mesh to act as an Avatar for the simulated cloth mesh.
\end{itemize}

\section{Human Avatar}
\label{section_human_avatar}


% Although I have experience with modeling in Blender, designing a 3-D human avatar from scratch seemed unnecessary, 
% as it is not in my area of expertise and there are available models online. 
% After a vast search, I decided to utilize the model I have acquired from a forum, 
% as it was realistic in appearance, proportions and details \cite{Mmava2012}. 
% In order to make it look more realistic, I needed to implement additional features.

After taking the main topic of the research, complexity of both 3D CAD programs and the human body itself into consideration, 
designing a complete full human body mesh was deemed to be a laborious and inessential task. Through researching various digital sources, 
evaluating the quality and cost of various potential candidate meshes, a female\cite{Mmava2012} model and a male model\cite{Gomer2013} were elected, due to their 
realistic appearence, accurate proportions and fine details. For the purpose of improving the realism of the simulation, details are introduced to the base model,
 with the Blender CAD software\cite{Blender}.  

\subsection{Rigging}

Animating a 3D mesh requires skinning, which is moving the vertices with respect to a bone on a skeleton. The base models had no skinning or material information, they required
custom rigging and painting. For the purposes of this research a manual rigging and painting would be preferred, in order to be able to integrate the model easily into the software. 

The rigging is performed in Blender\cite{Blender}, where the base human skeleton is provided, detailed in H-ANIM2 level with a much simplified spine 
(Figure \ref{fig:rigging_skeleton}). 

\begin{figure}[h]
\centerline{\fbox{\psfig{figure=figures/rigging_skeleton.png,width=1.00\textwidth}}}
\caption{The Rigging Base Skeleton}
\label{fig:rigging_skeleton}
\end{figure}

Prior to rigging, the skeleton needs to be in perfect alignment with the mesh. After modifying the initial bone positions, 
the vertex groups are assigned to the bones with proper weights (Figure \ref{fig:weight_humerus_r}).
 After assigning every vertex to the appropriate bones, there were no cracked surfaces with motions which are natural for humans. 

In the end, the models were exported in OGRE .mesh format along with the skeleton it used.

\begin{figure}[h]
\centerline{\fbox{\psfig{figure=figures/weight_humerus_r.png,width=1.00\textwidth}}}
\caption{The vertex weights for the Humerus.R bone}
\label{fig:weight_humerus_r}
\end{figure}

\subsection{Materials}

The base model came with no material information. In order to improve the realism of the model, texturing/painting tasks for various parts of the body were performed.
 These parts include the general skin, hair, eyes and lips (Figure \ref{fig:detailed_face}). Other additions include accessories such as hair sticks and earrings
 to achieve more realistic results.

\begin{figure}[h]
\centerline{\fbox{\psfig{figure=figures/detailed_face.png,width=1.00\textwidth}}}
\caption{Detailed Materials on the Face}
\label{fig:detailed_face}
\end{figure}

\section{Cloth Mesh}
\label{section_cloth_mesh}

As one of the major pieces of this study is accurate cloth simulation, accurately modeled cloth meshes which are suitable for simulation for required. 
After my searching and filtering through a variety of models online, a set of base apparel meshes for both male and female models are selected \cite{LadyJewell2012,
3dregenerator2013,Axel2013,Borodin2013,PS3D2013,Alperin2013}. The models vary in detail and quality, all of them were optimized and enhanced for the framework. 

\subsection{Body Positioning and Splitting the Dress Mesh}

For a successful animation and simulation of the dress on a human avatar, both meshes must be in proportion with each other and properly aligned. 
The required modifications to the meshes were performed in Blender(Figure \ref{fig:dress_and_body}).

The initial approach was to include all the vertices of the clothing mesh in the simulation. However after various attempts to simulate all the vertices on the dress mesh,
 this approach failed to achieve a realistic looking results due to two main reasons:
\begin{itemize}
  \item The whole dress consisted of 4088 vertices. The simulation was running in real-time, however the topography of the clothing mesh was not appropriate for the physics engine due 
  to the very large number of vertices in the cloth. It shifted from a fabric structure to more of a jelly form, over stretching and tearing with its own weight.
  \item The friction necessary to keep the dress on the Human Avatars shoulders was not sufficient enough to keep the dress on the avatar. It kept sliding, stretching and acting unnaturally.
\end{itemize}

In order to overcome these problems, the dress mesh is split into two parts, and utilize different animation techniques on each one.

\begin{itemize}
  \item The static part, which should be attached to the body, not blowing in the wind, was modeled as a static mesh, with skinning like the human avatar. 
  It was animated with the same transformations as the human avatar, matching its position and staying on the body perfectly.
  \item The dynamic part was the part to be simulated, affected by inertia, wind and other factors. Their attachment line is just above the waist
   of the human avatar, which was pinpointed through try-and-error experimentations with other locations (Figure \ref{fig:dress_and_body}). 
   Keeping the line too below resulted in both unnatural collisions and the cloth being too rigid. Keeping it too high brought out the problems in the first approach.
    With its current setting, the dress mesh is animated naturally on the virtual avatar.
\end{itemize}

\begin{figure}[h]
\centerline{\fbox{\psfig{figure=figures/dress_and_body.png,width=1.00\textwidth}}}
\caption{The  Dress, positioned on the body, along with the upper-part skeleton. In this shot, the dynamic part is highlighted in orange border.}
\label{fig:dress_and_body}
\end{figure}

