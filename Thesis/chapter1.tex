\chapter{Introduction}
\label{chapter1}

A mesh is a collection of polygonal facets targeting to constitute an appropriate approximation of a real 3D object~\cite{Wang07}. It has vertices, edges and facets. A mesh stores two kinds of information: \textit{geometry} and \textit{connectivity}. Geometry information gives the position of its vertices and the connectivity information gives the relationship between its elements. Mesh deformation means modifying the shape of the original object by using control points or external forces. 3D mesh deformation has been a highly active research area in computer graphics. Deformations have widespread usage areas like computer games, computer animations, fluid flow, heat transfer, surgical simulation, cloth simulation, crash test simulations. The major goal in mesh deformations is to establish a good balance between the accuracy of the simulation and the computational cost, which depends on the application.

In order to make deformations realistic and highly accurate, Finite Element Method (FEM) could be used. FEM is a numerical method to find approximate solutions to the problems of engineering and mathematical physics. Typical problems that are solvable by use of the finite element method include structural analysis, heat transfer, fluid flow, mass transport, and electromagnetic potential~\cite{Logan07}.

It is not possible to obtain an analytical solution for problems involving complicated geometries (e.g., 3D organ models), loadings, and material properties. Analytical solutions generally require the solution of ordinary or partial differential equations. However, they are not generally obtainable because of the problems mentioned above~\cite{Logan07}. When we incorporate these into the analytical solution, the degree of the partial differential equations becomes so high that the solution is not obtainable. Numerical methods that find approximate solutions are used to overcome this problem. Among these, Rayleigh-Ritz, Galerkin, and Finite Difference methods, are the most common ones. Finite difference method approximates the differential equations with equivalent difference quotients using limits over the domain. In other words, the method approximates the solution of differential equations by using approximations to the derivatives. Rayleigh-Ritz method introduces trial functions like FEM's weight functions. However, it only works for conservative systems and it gets very complicated for complex geometries, such as 3D organ meshes, crash test models. Galerkin method uses weighted residuals to calculate a global stiffness matrix, as FEM does. However, none of these techniques can handle geometrically complex domains. The approximation to the solution becomes very complicated even in simpler domains.

FEM simplifies these calculations by subdividing the given domain into a finite set of subdomains, called \textit{finite elements}~\cite{Reddy93}. The problem becomes easier and solvable over these domains, such as a set of rectangles, triangles, and in our case, tetrahedra. When using FEM for all types of problems; 1D, 2D, 3D, linear or nonlinear, it follows certain steps; finite element discretization, derivation of the element displacement functions, assembly of the elements, imposition of the boundary conditions and solving the system to find an approximate solution of the given domain~\cite{Reddy93}.

Surgical simulations require high accuracy, very low error tolerance and real-time interaction that cannot be fully handled with popular deformation techniques, such as regular deformations, Free-form deformation and mass spring system. Numerical solutions that require high accuracy must be used such as finite difference, variational methods and FEM. By far the most popular numerical solution technique is FEM with surgical simulations. Firstly, linear FEM models are used due to faster calculation compared to nonlinear FEM. Cover et al. used surface models to model the system~\cite{Cover93}. However, surface models are not sufficient for surgical simulators to perform internal surgical operations. Along with the increase in the computational power of the computers, the work of Bro-nielsen prompted the usage of volumetric models~\cite{Bro96}. Cavusoglu used nonlinear model's reduction to linear model using linear FEM~\cite{Cavusoglu00}. Most recent surgical simulators use nonlinear FEM~\cite{Taylor07,Horton2010} because nonlinear FEM is capable of providing realistic predictions of finite deformations of the tissue. Although these recent systems provide real-time simulations, the preprocessing step to calculate the stiffness matrices still takes many hours.

\section{Contributions}

We implemented linear FEM and non-linear FEM by using tetrahedral elements. We used same material properties (constitutive matrices) with linear and nonlinear FEM. In this way, we are able to observe the effect of using the same material with nonlinear geometric properties and compare the results with the results of linear FEM. The contributions can be summarized as follows:


\begin{enumerate}
\item We propose a new formulation and finite element solution to the nonlinear 3D elasticity theory.
\item We derived nonlinear stiffness matrices by using the Green-Lagrange strains (large deformation). Green-Lagrange strains are derived directly from the infinitesimal strains (small deformation) by adding the nonlinear terms that are discarded in infinitesimal strain theory.
\item We propose a more comprehensible nonlinear FEM for whom has knowledge about linear FEM since proposed method directly derived from the infinitesimal strains.
\item We use geometric nonlinearity to compare the results of nonlinear FEM with the results of linear FEM.
\item We compare our approach with Pedersen's method to measure the performance and verify the correctness. We achieve $111\%$ speed-up for the calculation of stiffness matrices and $17\%$ on average for the solution of the whole system, compared to the Pedersen's method.
\end{enumerate}



\section{Overview of the Thesis}

The rest of the thesis is organized as follows. Chapter~\ref{chapter2} describes the related work on deformation techniques, Finite Element Method, and the nonlinear analysis. Chapter~\ref{chapter3} discusses variational approaches, such as Galerkin, Rayleigh-Ritz and Weighted-Residual methods, and finite difference methods. Chapter~\ref{chapter4} explains linear FEM. Chapter~\ref{chapter5} describes the nonlinear FEM, including a detailed discussion of the Pedersen's method and the proposed approach. Chapter~\ref{chapter6} gives the experimental results obtained using the proposed approach and compares the computational overhead of the proposed approach with that of Pedersen. Chapter~\ref{chapter7} gives conclusions and possible future research directions.
