\chapter{Appendix}
\label{chapter8}

We conducted two different experiments to extend our solution in the area of rhinoplasty. In the two experiments, our aim is to deform the misshapen noses into a perfect nose (see~Figure~\ref{fig:app1}), and our aim in these experiments is also to compare the accuracy of the deformations for linear and nonlinear FEMs. The results for these experiments are interpreted by comparing displacement amounts for the force applied nodes and all the nodes. We did not compare the computational costs or strain-displacement relationships in this chapter, because we thoroughly examined these issues in Chapter~\ref{chapter6}.

\begin{figure}[c]
\centerline{\fbox{\psfig{figure=figures/app1.ps,width=0.95\textwidth}}}
\caption{The perfect nose.}
\label{fig:app1}
\end{figure}

\section{Experiment 1}
\label{ai}

The first experiment is conducted with a head mesh that has 6709 nodes and 25722 tetrahedral elements (see~Figure~\ref{fig:app2-1}). Number of tetrahedral elements are very high. However, all the operations are done in the nose area and there are 1458 tetrahedral elements in this area. To ease up the calculations, stationary tetrahedral elements are canceled. Figure~\ref{fig:app2} shows that the head mesh is constrained from the blue nodes, and is pushed upwards at the green nodes. This experiment is conducted to observe the different effects of the nonlinear and the linear FEM deformations over the nose. Linear FEM produced high amount of displacement at the upper part of the nose(see~Figure~\ref{fig:app3}). As a result of that, the node displacement error becomes 64.92\%. It can be observed that the nose was deformed more realistically and smoothly with nonlinear FEM (see~Figure~\ref{fig:app4}). The nose is more collapsed inwards with linear FEM, whereas its structure is better preserved with nonlinear FEM and the overall shape is more similar to a perfect nose than linear FEM. Although, nearly 6000 nodes are constrained, the overall nodal displacement error is 3.88\%.

\begin{figure}[h]
\centerline{\fbox{\psfig{figure=figures/app2-1.ps,width=0.95\textwidth}}}
\caption{Experiment 1: Shaded head mesh with texture.}
\label{fig:app2-1}
\end{figure}

\begin{figure}[h]
\centerline{\fbox{\psfig{figure=figures/app2.ps,width=0.95\textwidth}}}
\caption{Experiment 1: Head mesh is constrained from the blue nodes, and is pushed upwards at the green nodes.}
\label{fig:app2}
\end{figure}

\begin{figure}[h]
\centerline{\fbox{\psfig{figure=figures/app3.ps,width=0.95\textwidth}}}
\caption{Linear FEM Solution: (a)~left: wireframe surface mesh with nodes, right: shaded mesh with texture.}
\label{fig:app3}
\end{figure}

\begin{figure}[h]
\centerline{\fbox{\psfig{figure=figures/app4.ps,width=0.95\textwidth}}}
\caption{Nonlinear FEM Solution: (a)~left: wireframe surface mesh with nodes, right: shaded mesh with texture.}
\label{fig:app4}
\end{figure}

\clearpage
\section{Experiment 2}
\label{aii}

The second experiment is conducted with similar mesh that is used in the first experiment. However, it has 7071 nodes and 27020 tetrahedral elements due to different shape of the nose and tetrahedralization (see~Figure~\ref{fig:app5}). To ease up the calculations, stationary tetrahedral elements are also canceled. Calculations are done with 4511 tetrahedra. Figure~\ref{fig:app6} shows that the head mesh is constrained from the blue nodes, and is pushed upwards at the green nodes. This experiment is conducted to observe the different effects of the nonlinear and the linear FEM deformations over the nose. Linear FEM produced high amount of displacement at the lower part of the nose (see~Figure~\ref{fig:app7}). Moreover, the nose is nearly collapsed inwards that is far away from the perfect nose. As a result of that, the node displacement error becomes 96.03\%. With nonlinear FEM, the nose is deformed more realistically and smoothly (see~Figure~\ref{fig:app8}). The nose is more collapsed inwards with linear FEM, whereas its structure is better preserved with nonlinear FEM and the overall shape is more similar to a perfect nose than linear FEM. Although, nearly 5000 nodes are constrained, the overall nodal displacement error is 11.19\%.

\begin{figure}[h]
\centerline{\fbox{\psfig{figure=figures/app5.ps,width=0.95\textwidth}}}
\caption{Experiment 2: Shaded head mesh with texture.}
\label{fig:app5}
\end{figure}

\begin{figure}[h]
\centerline{\fbox{\psfig{figure=figures/app6.ps,width=0.95\textwidth}}}
\caption{Experiment 2: Head mesh is constrained from the blue nodes, and is pushed upwards at the green nodes.}
\label{fig:app6}
\end{figure}

\begin{figure}[h]
\centerline{\fbox{\psfig{figure=figures/app7.ps,width=0.95\textwidth}}}
\caption{Linear FEM Solution: (a)~left: wireframe surface mesh with nodes, right: shaded mesh with texture.}
\label{fig:app7}
\end{figure}

\begin{figure}[h]
\centerline{\fbox{\psfig{figure=figures/app8.ps,width=0.95\textwidth}}}
\caption{Nonlinear FEM Solution: (a)~left: wireframe surface mesh with nodes, right: shaded mesh with texture.}
\label{fig:app8}
\end{figure}
