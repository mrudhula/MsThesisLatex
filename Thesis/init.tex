\documentclass{buthesis}
\usepackage{graphicx}
\usepackage{epsfig}
\usepackage{booktabs}
\usepackage{multirow}
\usepackage{epstopdf}
\usepackage[linesnumbered,ruled,vlined]{algorithm2e}
\usepackage[breaklinks,linktocpage]{hyperref}  
\usepackage[hyphenbreaks]{breakurl}

\usepackage{amssymb,amsmath}
\usepackage{pdflscape} 

\pagenumbering{roman}

\thesistype{M.S.~}
\teztipi{Y{\"u}ksek Lisans}

\keywords{cloth simulation, computer vision, natural interaction, virtual fitting room, kinect, depth sensor}
\anahtarsoz{k{\i}yafet sim{\"u}lasyonu, bilgisayarla g{\"o}r{\"u}, do\u{g}al etkile\c{s}im, sanal giyinme kabini, kinect, derinlik sens{\"o}r{\"u}}

\title{REAL TIME PHYSICS-BASED AUGMENTED FITTING ROOM USING TIME-OF-FLIGHT CAMERAS}
\baslik{U\c{C}U\c{S} ZAMANI KAMERALARI KULLANAN \\ \vspace*{-1.25ex} GER\c{C}EK ZAMANLI F\.{I}Z\.{I}K TABANLI \\ARTTIRILMI\c{S} G\.{I}Y\.{I}NME KAB\.{I}N\.{I}}

\author{Umut G{\"u}ltepe}
\dept{Computer Engineering}
\bolum{Bilgisayar M\"{u}hendisli\u{g}i}
\principaladvisor{Assoc. Prof. Dr. U\u{g}ur G\"{u}d\"{u}kbay}
\tezyoneticisi{Do\c{c}. Dr. U\u{g}ur G\"{u}d\"{u}kbay}
\firstreader{Prof. Dr. \"{O}zg\"{u}r Ulusoy}
\secondreader{Asst. Prof. Dr. Ahmet O\u{g}uz Aky\"{u}z}
\director{Prof. Dr. Levent Onural}
%\copyrightyear{2012}
\submitdate{July, 2013}
\tarih{Temmuz, 2013}


\begin{document}
\titlepageMS
%\setcounter{page}{2}
\signaturepageMS
%\setlength{\baselineskip}{6mm}
%\thispagestyle{plain}
\begin{abstract}
\indent This thesis proposes a framework for a real-time physically-based augmented cloth fitting environment. The required 3D meshes for the human avatar and apparels are modeled with specific constraints. The models are then animated in real-time using input from a user tracked by a depth sensor. A set of motion filters are introduced in order to improve the quality of the simulation. The physical effects such as inertia, external and forces and collision are imposed on the apparel meshes. The avatar and the apparels can be customized according to the user. The system runs in real-time on a high-end consumer PC with realistic rendering results.
\end{abstract}


\begin{ozet}
\indent Bu tezde ger\c{c}ek zamanl{\i} fizik tabanl{\i} bir art{\i}r{\i}lm{\i}\c{s} giyinme kabini ortam{\i} i\c{c}in bir \c{c}al{\i}\c{s}ma \c{c}er\c{c}evesi {\"o}nerilmektedir. {\.I}nsan avatar{\i} ve k{\i}yafet i\c{c}in gerekli {\"u}\c{c} boyutlu modeller {\"o}zel s{\i}n{\i}rlar \c{c}er\c{c}evesinde modellenmi\c{s}tir. Bu modeller daha sonra bir derinlik al{\i}c{\i}s{\i} taraf{\i}ndan takip edilen bir kullan{\i}c{\i}dan al{\i}nan girdi ile ger\c{c}ek zamanl{\i} olarak hareket ettirilmektedir. Sim{\"u}lasyon kalitesini artt{\i}rmak amac{\i} ile hareketler \c{c}e\c{s}itli filtrelerden ge\c{c}irilmektedir.  Eylemsizlik, d{\i}\c{s} kuvvetler ve \c{c}arp{\i}\c{s}ma gibi d{\i}\c{s} etkenler k{\i}yafet modeline uygulanmaktad{\i}r. Avatar ve k{\i}yafet modelleri, kullan{\i}c{\i}n{\i}n boyutlar{\i}na g{\"o}re {\"o}zelle\c{s}tirilebilir. Sistem {\"u}st kalite bir ki\c{s}isel bilgisayar {\"u}zerinde ger\c{c}ek zamanl{\i} olarak ger\c{c}ek\c{c}i g{\"o}r{\"u}nt{\"u}ler olu\c{s}turarak \c{c}al{\i}\c{s}maktad{\i}r. 
\end{ozet}

\begin{ack}

I would like to express my sincere gratitude to my supervisor Assoc. Prof. Dr. U\u{g}ur G{\"u}d{\"u}kbay, who guided and assisted me with his invaluable suggestions in all stages of this study. I also chose this area of study by inspiring from his deep knowledge over this subject.

I am very grateful to my jury members Prof. Dr. \"{O}zg\"{u}r Ulusoy and Asst. Prof. Dr. Ahmet O\u{g}uz Aky\"{u}z for reading and reviewing this thesis. 

I would like to thank Computer Engineering Department of Bilkent University for providing me scholarship for my MS study. I also would like to thank the Scientific and Technical Research Council of Turkey (T{\"U}B\.{I}TAK) and the Turkish Ministry of Industry and Technology for their financial support for this study and MS thesis.

\end{ack}

\newpage
\setcounter{page}{5}
\vspace*{5cm}
\begin{center}
{\large \it to my mother, father and my brother...}
\end{center}

\tableofcontents
\listoffigures
\listoftables
\listofalgorithms
\newpage
\newpage
%\input{symbols}
\newpage
\pagestyle{headings}
\makeatother
