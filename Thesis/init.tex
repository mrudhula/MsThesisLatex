\documentclass{buthesis}
\usepackage{graphicx}
\usepackage{epsfig}
\usepackage{booktabs}
\usepackage{multirow}
\usepackage{epstopdf}
\usepackage[linesnumbered,ruled,vlined]{algorithm2e}

\usepackage{amssymb,amsmath}
\usepackage{pdflscape}

\pagenumbering{roman}

\thesistype{M.S.~}
\teztipi{Y{\"u}ksek Lisans}

\keywords{cloth simluation, computer vision, natural interaction, virtual fitting room, kinect, depth sensor}
\anahtarsoz{k{\i}yafet sim{\"u}lasyonu, bilgisayarla g{\"o}r{\"u}, do\u{g}al etkile\c{s}im, sanal giyinme kabini, kinect, derinlik sens{\"o}r{\"u}}

\title{AUGMENTED CLOTH FITTING WITH REAL TIME PHYSICS SIMULATION USING TIME-OF-FLIGHT CAMERAS}
\baslik{U\c{C}U\c{S} ZAMANI KAMERALARI KULLANAN GER\c{C}EK ZAMANLI F\.{I}Z\.{I}K S\.{I}M{\"U}LASYONLU ARTIRILMI\c{S} KIYAFET KAB\.{I}N\.{I}}

\author{Umut G{\"u}ltepe}
\dept{Computer Engineering}
\bolum{Bilgisayar M\"{u}hendisli\u{g}i}
\principaladvisor{Assoc. Prof. Dr. U\u{g}ur G{\"u}d{\"u}kbay~}
\tezyoneticisi{Do\c{c}. Dr. U\u{g}ur G{\"u}d{\"u}kbay~}
\director{Prof. Dr. Levent Onural}
%\copyrightyear{2011}
\submitdate{December, 2012}
\tarih{Aral{\i}k, 2012}


\begin{document}
\titlepageMS
%\setcounter{page}{2}
\signaturepageMS
%\setlength{\baselineskip}{6mm}
%\thispagestyle{plain}
\begin{abstract}
\indent This study surveys and proposes a method for an augmented cloth fitting with real time physics simulation. Augmented reality is an evolving field in computer science, finding many uses in entertainment and advertising. With the advances in cloth simulation and time-of-flight cameras, augmented cloth fitting in real-time is developed , to be used in textile industry in both design and sale stages. Delay in cloth fitting due to processing time is the main challenge in this research. Human body is identified, articulated and tracked with a time-of-flight camera. Depending on the size and position of body limbs, a virtually simulated cloth is fitted in real time on the subject. Delay is reduced with GPU computing for cloth simulation and collision detection.
\end{abstract}



\begin{ozet}
\indent Bu \c{c}al{\i}\c{s}ma fizik sim{\"u}lasyonlu bir art{\i}r{\i}lm{\i}\c{s} k{\i}yafet giydirme i\c{c}in bir metot {\"o}nermekte ve incelemektedir. Art{\i}r{\i}lm{\i}\c{s} ger\c{c}eklik bilgisayar biliminde geli\c{s}en bir aland{\i}r, e\u{g}lence ve reklam sekt{\"o}rlerinde geni\c{s} yer bulmaktad{\i}rlar. Kuma\c{s} sim{\"u}lasyonu ve u\c{c}u\c{s}-zaman{\i} kameralar{\i}n{\i}n geli\c{s}tirilmesi ile, ger\c{c}ek zamanl{\i} art{\i}r{\i}lm{\i}\c{s} k{\i}yafet giydirimi tekstil end{\"u}strisinde tasar{\i}m ve sat{\i}\c{s} a\c{s}amalar{\i}nda kullan{\i}lmak {\"u}zere geli\c{s}tirilmi\c{s}tir. �\c{s}leme zaman{\i} sebebiyle k{\i}yafet giydirilmesindeki gecikme bu ara\c{s}t{\i}rmadaki en b{\"u}y{\"u}k zorluktur. �nsan v{\"u}cudu bir u\c{c}u\c{s}-zaman{\i} kameras{\i} ile tan{\i}mlanmakta, b{\"o}l{\"u}nmekte ve takip edilmektedir. V{\"u}cut par\c{c}alar{\i}n{\i}n boyutlar{\i}na ve pozisyonlar{\i}na g{\"o}re sim{\"u}le edilen bir sanal k{\i}yafet kullan{\i}c{\i}n{\i}n {\"u}st{\"u}ne yerle\c{s}tirilmektedir. Gecikme zaman{\i} k{\i}yafet sim{\"u}lasyonunu ve \c{c}arp{\i}\c{s}ma takibini GPU {\"u}zerinde yaparak azalt{\i}lmaktad{\i}r.  
\end{ozet}

\begin{ack}

I am deeply grateful to my supervisor Assoc. Prof. Dr. U\u{g}ur G{\"u}d{\"u}kbay, who guided and assisted me with his invaluable suggestions in all stages of this study. I also chose this area of study by inspiring from his deep knowledge over this subject.

I am very grateful to Computer Engineering Department of Bilkent University for providing me scholarship for my MS study. 

I would like to thank Scientific and Technical Research Council of Turkey (T{\"U}B\.{I}TAK) and the Turkish Ministry of Industry and Technology for their financial support for this study and MS thesis.

\end{ack}

\newpage
\setcounter{page}{5}
\vspace*{5cm}
\begin{center}
{\large \it to my mother, father and my brother...}
\end{center}

\tableofcontents
\listoffigures
\listoftables
\listofalgorithms
\newpage
\newpage
%\input{symbols}
\newpage
\pagestyle{headings}
\makeatother
